\chapter{Anforderungsanalyse} 
\label{Kapitel 3}
\lhead{Kapitel 3. \emph{Anforderungsanalyse}}  
Detaillierte Untersuchung des Problembereichs
Theoretische oder praktische Herleitung der Anforderungen
Zusammenfassung der Anforderungen (bspw. als Tabelle)

\section{Use Case}
Der folgende Use-Case soll am Ende der Masterarbeit idealerweise abgedeckt werden:

\subsection{Ziel}
Betreuung eines Studienkurses über das Internet

\subsection{Voraussetzungen}
Kursbetreuer/Mentor sitzt in Deutschland, die Studenten in Kamerun. Es steht dort nicht immer ein
Internetzugang zur Verfügung. Des weiteren wird oft nach Zeit und nicht nach Volumen
abgerechnet, so dass der Nutzer auch bei dem Vorhandensein eines Internetzugangs nicht zwingend
online sein muss. Durch die Arbeit an unterschiedlichen Rechnern mit potentiell unterschiedlichen
Betriebssystemen, ist die Installation einer komplexen Software nicht ohne Weiteres möglich.
Workflow:
Betreuer und Studenten stehen über unterschiedliche Kanäle in Kommunikation miteinander. Es
müssen Termine geplant, Notizen und Nachrichten hin und hergeschickt werden. Dabei sind die
Teilnehmer zu unterschiedlichen Zeiten online. Die Teilnehmer sollen das System offline Nutzen
können, um einfache Arbeiten wie das Schreiben von Twitter-Nachrichten, Notizen und Instant-
Messaging Nachrichten oder eine Terminabsprache über einen Kalender erledigen können. Bei dem
Wechsel zwischen Online und Offline müssen die Daten synchronisiert werden. Idealerweise haben
die Nutzer alle Daten auf einem USB-Stick bei sich und können so von unterschiedlichsten
Rechnern, wie beispielsweise in der Universität, im Internetcafe oder zu von Hause aus, arbeiten.
Konzeption:
Die Arbeit wird sich in die folgenden zwei Bereiche aufteilen:
\begin{enumerate}

\item Konzeption, Design und Implementierung eines leichtgewichtigen auf Html5 und Javascript
basierenden Dashboards.

\item Konzeption und prototypische Implementierung eines oder mehrerer Workflows, die es
erlauben mit den Widgets zu kommunizieren und zumindest Teile der Services zu nutzen,
auch wenn das System (also der Browser) offline ist. Wird die Internetverbindung
wiederhergestellt sollen vorgenommene Änderungen und Arbeiten mit dem Onlineservice
synchronisiert werden.

\end{enumerate}
