\chapter{Übersicht Stand der Forschung/Technik} 
\label{Kapitel 4}
\lhead{Kapitel 4. \emph{Übersicht Stand der Forschung}} 
Darstellung existierender Ansätze (ggf. Klassifikation dieser Ansätze)
Analyse der Ansätze in Bezug auf die Anforderungen
Zusammenfassung der Defizite des Stands der Forschung

E-Learning E-Learning 1.0 - 2.0 
von kurs zentriert zu user zentriert

E-learning is a technological infrastructure with applications and software that manage courses
and users. The software that facilitates e-learning may be called a Learning Management System (LMS) and supports course creation, content delivery, user registration, monitoring, and certification.

Gonella and Panto (2008), in their paper on didactic architectures, have traced the following four
stages in the evolution of e-learning:
1. Web-based Training 2. E-learning 1.0 3. Online Education 4. E-learning 2.0


\section{PLE}

\subsection{Definition einer PLE}
Was ist eine PLE? Worin unterscheidet sie sich von anderen Lernumgebungen? Welche Ansätze gibt es? 

\subsection{Klassifizierungsmethoden}
Es existieren zwei wichtige Ansätze um Funktionalitäten von PLEs auf unterschiedliche Klassen abzubilden. Zum einen gibt es den Versuch von Palmér Dimensionen zu definieren und die Funktionalitäten von PLEs diesen Dimensionen zuzuordnen. Palmér schafft damit ein System, um unterschiedliche PLE-Systeme kategorisieren zu können  Zum anderen definiert Wilson Muster, die helfen sollen eine PLE zu entwickeln und bestehende Systeme auf wichtige Funktionalitäten zu überprüfen. Diese beiden Ansätze werden im Folgenden vorgestellt .

\subsubsection{Dimensionen nach Palmér} 
Palmér definiert sechs Dimensionen mit denen er so viele relevanten Funktionalitäten von PLEs wie möglich erfassen möchte \cite{Palmer2009}. Trotz dessen sollen diese Dimensionen relativ unabhängig voneinander sein, so dass es möglich ist das unterschiedliche Plattformen einige Dimensionen mehr und andere weniger berücksichtigen und implementieren. (TODO Grafik der Dimensionen). Eine PLE kann dann Anhand des Grades ihrer Implementierung der einzelnen Dimensionen kategorisiert und bewertet werden.
\begin{enumerate}
 \item \emph{Screen-Dimension}: Die Screen-Dimension befasst sich mit Aspekten, welche die Darstellungsebene von PLEs definieren. Hierzu zählt Palmér insbesondere das User-Interface und die Usability des PLE-Containers (wie sind die Widgets angeordnet, wie können neue Widgets gesucht und hinzugefügt werden, wie einfach kann sich der Nutzer im System bewegen etc.), welcher als Einstiegspunkt in die Systembedienung dient und das User-Interface der einzelnen Widgets. Es gehören aber auch Funktionalitäten wie die Möglichkeit Inhalte und Ressourcen mit anderen Nutzern zu teilen und die Integration der selben Widgets in unterschiedlichen PLE-Containern zu der Screen-Dimension.
 \item \emph{Data-Dimension}: Mit der Data-Dimension beschreibt Palmér Funktionalitäten, die für die Portabilität der verwendeten Daten innerhalb einer PLE notwendig sind. Idealerweise sollen Widgets in der Lage sein untereinander und mit dem PLE-Container zu kommunizieren. Sie sollen Daten austauschen können und sich so weit wie möglich über ihre Zustände informieren. Des Weiteren soll es möglich sein die Daten der Widgets zu exportieren und sie an anderer Stelle oder in einem anderen PLE-Container wieder zu importieren und weiterzuverwenden. Mit der fortschreitenden Mobilität der Nutzer wird es immer wichtiger, dass der Zugriff auf die PLE auch dann möglich ist, wenn kein Zugriff auf das Internet besteht. Somit ist es nicht nur notwendig Daten zu importieren und zu exportieren, sondern auch zwischen einem Offline und einem Online-Speicher zu synchronisieren, sobald eine entsprechende Zustandsänderung eintritt.
 \item \emph{Social-Dimension}: Ein wesentlicher Bestandteil des Web 2.0 ist die Vernetzung von Freunden und Menschen mit ähnlichen Interessen untereinander. Dem trägt Palmér mit der Social-Dimension Rechnung. Diese Dimension gibt an, wie sehr eine PLE Funktionalitäten sozialer Netzwerke wie Freundeslisten integriert und Möglichkeitkeiten bietet den Zugriff auf geteilte Ressourcen auf bestimmte Typen von Freunden einzuschränken. Palmér zählt aber auch die Möglichkeit zur Erstellung von eigenen offenen oder geschlossenen Lerngruppen zu Funktionalitäten, die in diese Dimension fallen.
 \item \emph{Temporal-Dimension}: Die in der Screen- und der Social-Dimension beschriebene Kollaboration zwischen Nutzern der PLE neue Anforderungen mit sich. So ist es beispielsweise notwendig oder zumindest wünschenswert, dass geänderte Inhalte sich in Echtzeit in den Instanzen der Widgets manifestieren, die ebenfalls auf diese Inhalte zugreifen. Hierbei sollten auch Probleme, wie auftretende Konflikte bei gleichzeitigem Bearbeiten der selben Ressource in Betracht gezogen werden.
 \item \emph{Activity-Dimension}: Die Activity-Dimension beschreibt die Möglichkeit Abläufe und Workflows innerhalb einer PLE aktiv zu gestalten. Hierzu gehören unterer anderem einfache Dinge wie Anleitungen als Hilfestellung für den Nutzer zur Bewegung innerhalb der PLE. Besonderen Wert legt Palmér aber auf die Abbildung von Lernsequenzen innerhalb der PLE. So können bestimmte Widgets auf bestimmte Ereignisse reagieren oder sich selbst aktivieren oder deaktivieren. Des Weiteren können unterschiedlichste Konzepte aus dem Bereich des E-Learnings (IMS Learning Desing (TODO cite) oder Learning Object Metadata (LOM)) in den Widgets oder dem PLE-Container selber implementiert werden.
 \item \emph{Runtime-Dimension}: Die Runtime-Dimension befasst sich mit Funktionalitäten, die die Interoperabilität zwischen PLE-Systemen und Komponenten. Nach Palmér werden Nutzer in der Zukunft nicht nur eine PLE benutzen, sondern je nach Bedürfnis oder Anwendungsfall zwischen ihnen hin und herwerchseln. Hierfür sollte es möglich sein Importe und Exporte für Inhalte und Einstellungen sowohl der einzelnen Widgets als auch des PLE-Containers vorzunehmen. Damit eine Interoperabilität
 zwischen PLEs möglich wird, ist es notwendig, dass Standards geschaffen werden, welche von den unterschiedlichen PLEs anerkannt und implementiert werden. Zusätzlich gehört für Palmér auch die Möglichkeit der Einbettung und Kommunikation von PLEs in und mit größeren Systemen zu dieser Dimension.
\end{enumerate}

\subsubsection{Wilson Patterns}
\cite{Wilson2008}
Einführung Wilson, warum Patterns, Patterns bei Softwareentwicklung Architektur, 77 Patterns, zu viele, Einschränkung, natürliche Patterns auf Beobachtung

Wilson unterscheidet zwischen zwei verschiedenen Arten von Mustern für PlEs: Muster für persönliche Anwendungen (Personal Tools) und Muster für Lernnetzwerke (Learning Networks). Personal Tools stellen hiebei die Werkzeuge dar, die ein Nutzer direkt für seine Lernaktivitäten nutzt. Er interagiert mit unterschiedlichen sozialen Netzen (zum Beispiel Lernnetzwerken) oder verwendet die Tools anderweitig für seine persönliche Art des Lernens. 
Wilson definiert Learning Networks als die Infrastruktur, welche notwendig ist um soziale Netze oder Communities aufzubauen. Des weiteren zählt er auch die Menge der Online Services hinzu, welche von einem Lehrinstitut den Lernenden zur Verfügung gestellt wird.

Diese Arbeit beschäftigt sich mit dem Aufbau einer PLE, welche direkt vom Nutzer für seine Lernaktivitäten verwendet wird. Learning Networks spielen hierfür keine oder eine nur sehr untergeordnete Rolle, so dass die Muster für das Erstellen eben dieser hier nicht weiter beleuchtet werden. Im Folgenden werden die Muster für Personal Tools vorgestellt. Diese ähneln meist einer Empfehlung, wie eine bestimmte Funktion umgesetzt werden sollte oder welche Funktionalitäten vorhanden sein sollten, um die Erfahrung des Nutzers im Umgang mit dem System zu verbessern.

\begin{enumerate}
 \item \emph{Discourse Monitor}: In einer PLE werden Informationen aus potentiell sehr vielen und unterschiedlichen Quellen verarbeitet. Um dem Nutzer die Möglichkeit zu geben wichtige Informationen schneller herauszufiltern oder zu prioritisieren, sollte ein Discourse Monitor implementiert werden. Dieser fasst die wichtigsten Informationen aus den unterschiedlichen Quellen zusammen und bereitet sie in übersichtlicher Art und Weise auf. Dem Nutzer soll es dann möglich sein, die dargestellten Daten zu Filtern, zu prioritisieren, seine Favoriten zu kennzeichnen oder neu eingetroffene Daten einfach und zeitnah zu erkennen. 
 \item \emph{Connection Hub}: Der Nutzer einer PLE ist bei verschiedensten Netzwerken angemeldet und hat dort höchstwahrscheinlich unterschiedliche Informationen hinterlegt. Es ist auch möglich, dass eine Kommunikation über Netzwerkgrenzen hinweg stattgefunden hat. Diese Daten sind meist nicht in einem einzigen Netzwerk darstellbar. Ein Connection Hub soll in der Lage sein oder den Nutzer in die Lage versetzen diese Verbindungen darzustellen und aufzubereiten. Es wäre beispielsweise vorstellbar, dass es dem Nutzer ermöglicht wird Informationen und Daten verschiedenster Netzwerke zu kombinieren ohne, dass diese Netzwerke direkt in Kontakt zueinander stehen.
 \item \emph{Create and Mix Media}: Es ist unabdingbar das während des computergestützen Lernens sowohl von dem Lehrnenden, als auch von dem Lehrenden Ressourcen erstellt und diese auch anderen zur Verfügung gestellt werden. Hierzu gehören alle Arten von Ressourcen, also Textdateien, Notizen, Quellcode, Videos, Bilder, Präsentationen etc.. Die PLE sollte es dem Nutzer ermöglichen diese Ressourcen zu erstellen (auch unter Zuhilfenahme externer Services) und diese dann an unterschiedliche Netzwerke zu verteilen. 
 \item \emph{Integrate Identities}: Durch das Nutzen unterschiedlicher Services hat der Nutzer höchstwahrscheinlich auch Useraccounts bei all diesen Services erstellt. Wilson schlägt für die Vereinfachung des Umgangs mit diesen Accounts die Implementierung von Mechanismen zur Vereinfachung vor. Hierzu gehören die Nutzung von Systemen, die in der Lage sind Zugangsdaten für unterschiedliche Accounts zu verwalten und zu speichern. Es können aber auch Konzepte wie das zentrale Hinterlegen von Profildaten, zum Beispiel bei einem OpenID Server-Anbieter \footnote{\url{http://openid.net/}}, verwendet werden. Die Registrierung und der Login bei unterschiedlichen Services erfolgt dann über den Umweg über den zentralen Service, welcher alle notwendigen Daten zu dem Service übermittelt, bei dem sich der Nutzer einloggen möchte.  
 \item \emph{Manage Time and Effort}: Benutzt man ein System wie eine PLE zur Organisation des persönlichen Lernens, so sollte einem dieses System Werkzeuge an die Hand geben, um das Lernen und die Arbeit zu organisieren. Hierzu gehören Zeitplaner, Kalender, Todo-Listen oder auch die Möglichkeit zur Erstellung von Notizen innerhalb des Systems.
 \item \emph{Navigation Layer}: Bei einer PLE hat sich der Nutzer idealerweise seine Lernumgebung aus unterschiedlichen Quellen und Werkzeugen zusammengestellt. Die Navigationsebene fasst diese in einem System zusammen und ermöglicht dem User einen einfachen Zugriff auf seine Werkzeuge. Wilson schlägt vor die Services als Widgets in der PLE einzubinden. Somit wird die PLE zu einem zentralen Zugriffspunkt oder zu einem Dashboard von dem aus der Nutzer alle seinen Aktionen ausführen kann. 
 \item \emph{Multi-platform/Multimode}: Die Nutzergewohnheiten bezüglich des Gebrauchs des Internets haben sich in den letzten Jahren deutlich gewandelt (TODO: schlaues cite). Der Zugriff erfolgt nicht mehr primär über den eigenen (Heim-)Rechner, sondern über die verschiedensten Zugriffspunkte. Hierzu gehören Computer in der Universität, am Arbeitsplatz, im Internetcafé und in letzter Zeit verstärkt auch mobile Geräte wie Smartphones oder Tablets. Aus diesem Grunde ist es notwendig, das der Zugriff auf Lernnetzwerke von all diesen Geräten aus möglich ist und die Systeme auch gut von den unterschiedlichen Geräten aus bedienbar sind. Des weiteren sollte es ein System geben, welches die Daten auf den unterschiedlichen Geräten miteinander synchronisier, so dass der Nutzer von überall auf die aktuellsten Daten zugreifen kann.
 \item \emph{Choose, Change, Discard}: Dieses Pattern steht in engem Bezug zu der nutzerzentrierten Herangehensweise in dem Aufbau von PLEs. User sollen in der Lage sein, sich ihre Lernumgebung nach eigenen Vorlieben und Anforderungen einzurichten. Außerdem ist es sehr gut möglich, dass sich die Anforderungen im Laufe der Zeit ändern. Aus diesen Gründen ist es notwendig, dass es dem Nutzer frei steht Inhalte und Services innerhalb der PLE zu verschieben und anzupassen und neue Werkzeuge hinzuzufügen und nicht mehr benötigte Werkzeuge wieder zu entfernen. Die PLE sollte hierbei nicht zu viele vom Nutzer nicht änderbare Bedienungsvorgaben und Konfigurationseinstellungen machen, sondern dem Nutzer so viele Freiheiten wie möglich bei Einrichten der eigenen Lernumgebung zu geben.
\end{enumerate}

Kritik an den Patterns?

\section{Technologie}
Im Folgenden werden Technologien und Konzepte vorgestellt, die den Stand der Technik im Bereich der PLE-Entwicklung darlegen oder notwendig sind, um die in Kapitel \ref{section:anforderungsanalyse} entwickelnden Anforderungen an eine Web basierte und offline fähige PLE zu erfüllen.


\section{Widgets}
Widgets
Opensocial vs W3C
warum w3c widgets, opensocial wird nicht weiterentwickelt etc, Apache Shindig Problematishc
W3C Widgets haben keine gesonderte Definition für HTML5 Appcache,
unterschiedliche Formate

Wookie während der Entwicklung vom Incubator zum vollwertigen Apache Projekt


\subsection{CORS}\label{section:cors}
Moderne Browser benutzen als Teil ihres Sicherheitskonzeptes die Same-Origin-Policy. Diese bewirkt, dass Sprachen, die auf Clientseite ausgeführt werden (wie JavaScript), nicht die Möglichkeit haben, Request an einen anderen Zielpunkt als ihrem Ursprung zu starten\cite{same_origin_policy_mozilla}. Diese Policy wird also lediglich bei Zugriff auf URLs mit der selben Domain und dem selben Port, wie die URL von der die Seite geladen wurde, erfüllt. Das bedeutet, dass ein Skript auf \texttt{http://sop.example.com/directory1} Requests an \texttt{http://sop.example.com/directory2} starten kann, nicht jedoch an \texttt{http://example.com/directory2} (unterschiedliche Domain) oder an \texttt{http://sop.example.com:8080/directory2} (unterschiedlicher Port). Ausgenommen ist hierbei das in eine Seite eingebettete Laden von Resourcen. Hierzu gehören externe Inhalte, die über iFrames geladen werden aber auch externe JavaScript-Dateien(über <script>...</script> Tags) und Medienresourcen wie Bilder und Videos. Des Weiteren ist es auch möglich Formulare an andere Zielpunkte als den Ursprung abzuschicken. Diese Einschränkung hat also primär Auswirkungen auf das Absenden von XMLHttpRequest, also auf normale Ajax-Requests. 

Die Same-Origin-Policy ist sehr sinnvoll um beispielsweise das Ausspähen privater Daten zu verhindern. Sie erschwert jedoch die Entwicklung moderner Ajax-Anwendungen und insbesondere die Entwicklung von Mashup-Applikationen wie PLEs, welche prinzipiell schon so aufgebaut sind, dass sie ihre Inhalte und Resourcen aus unterschiedlichen Quellen beziehen. In einer PLE wie in dieser Arbeit beschrieben, ist es beispielsweise so, dass die Widgets (siehe \ref{section:widgets}) selber von einem Widget-Container wie Wookie (siehe \ref{section:apache_wookie}) ausgeliefert werden und dadurch auch die Domain des Widget Containers als Origin besitzen. Arbeiten diese Widgets nun aber nicht nur lokal beim Client, sondern benötigen für ihre Funktionalität auch externe Server, so müssen sie in der Lage sein XMLHttpRequests an diese zu senden. Aus diesem Grund wurde der Mechanismus des Cross-Origin Resource Sharing (CORS)\cite{cors_w3c} eingeführt. Dieser erlaubt es unter bestimmten Bedingungen und Einschränkungen die Same-Origin-Policy zum umgehen.

Ein einfacher CORS-Request vom Client zum Server sieht wie folgt aus (Workflow analog zu \cite{cors_html5rocks}):

Der Client sendet eine Cross-Origin-Anfrage mit einem Origin Header an den Server:
\begin{lstlisting}
GET /cors HTTP/1.1
Origin: http://api.bob.com
Host: api.alice.com
Accept-Language: en-US
Connection: keep-alive
User-Agent: Mozilla/5.0...
\end{lstlisting}
 
Anschließend antwortet der Server mit:
\begin{lstlisting}
Access-Control-Allow-Origin: http://api.bob.com
Access-Control-Allow-Credentials: true
Access-Control-Expose-Headers: FooBar
Content-Type: text/html; charset=utf-8
\end{lstlisting}
Alle für den CORS-Request relevanten Header beginnen mit Access-Control. \texttt{Access-Control-Allow-Origin} bedeutet, dass der Server eine Cross-Origin-Anfrage von dem angegebenen Origin erlaubt, \texttt{Access-Control-Allow-Credentials: true} besagt, dass in diesem Request auch Cookies erlaubt sind. Möchte der Client Zugriff auf Nicht-Standard-Header aus der Antwort des Servers, müssen diese in \texttt{Access-Control-Expose-Headers} angegeben werden.

Sollte der Client einen Request mit einer anderen Methode als \texttt{GET} oder \texttt{POST} (siehe \ref{REST} senden, reicht dieser einfache Workflow nicht aus. In diesem Fall muss vor der eigentlichen Anfrage ein so genannter "`Preflight-Request"' ablaufen, welcher verifiziert, dass der Server diese Methode als CORS-Request erlaubt.

Zuerst wird vom Client eine Anfrage mit der \texttt{OPTIONS}-Methode durchgeführt, welche den folgenden Request authentifizieren soll:
\begin{lstlisting}
OPTIONS /cors HTTP/1.1
Origin: http://api.bob.com
Access-Control-Request-Method: PUT
Access-Control-Request-Headers: X-Custom-Header
Host: api.alice.com
Accept-Language: en-US
Connection: keep-alive
User-Agent: Mozilla/5.0...
\end{lstlisting}
\texttt{Access-Control-Request-Method} gibt hierbei an, welche Methode genutzt werden soll, \texttt{Access-Control-Request-Headers} informiert den Server über zusätzlich zu erwartende Header. 

Der Server antwortet beispielsweise mit:
\begin{lstlisting}
Access-Control-Allow-Origin: http://api.bob.com
Access-Control-Allow-Methods: GET, POST, PUT
Access-Control-Allow-Headers: X-Custom-Header
Content-Type: text/html; charset=utf-8
\end{lstlisting}
Der Preflight-Request ist nur erfolgreich, wenn die Methode aus \texttt{Access-Control-Request-Method} in \texttt{Access-Control-Allow-Methods} und alle Header aus \texttt{Access-Control-Request-Headers} in \texttt{Access-Control-Allow-Headers} vorhanden sind.

\subsection{HTML5}\label{section:html5}
neuerungen HTML5, nicht nur eine Technologie, Zusammenfassung mehrerer Technologien, CSS 3 unterschiedliche JavaScript Apis (local storage, appcache) drag and Drop

\subsubsection{Appcache}\label{section:appcache}
Eine wichtige Neuerung in Html

\subsubsection{Local Storage}\label{section:local_storage}
Local Storage


\subsection{REST}\label{section:rest}
Rest


\subsubsection{Widgetparameter}
Die Parameter für Widgets werden in der Wookie DB hinterlegt.
Die Widget Implementation wird über http://localhost:8080/wookie/shared/js/wookie-wrapper.js ausgeliefert.
Hier ist setItem/getItem ist so implementiert, dass es einen Request an den Server sendet und die Einstellungen speichert.

Die Standardeinstellungen aus der config.xml werden nur beim ersten deploy ausgelesen, anschließend nicht! mehr

Jedes Widget bekommt eine eigene id! Diese kann dann für die Identifizierung genutzt werden.

Die Werte werden NICHT im local storage hinterlegt.