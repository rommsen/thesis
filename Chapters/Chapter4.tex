\chapter{Übersicht Stand der Forschung / Technik}\label{chapter:Kapitel4}

Das vorliegende Kapitel stellt Technologie und Konzepte vor, welche den Stand der Technik und Forschung im Bereich der \ac{PLE}"=Entwicklung darstellen oder notwendig sind, um die in Kapitel \ref{chapter:Kapitel3} entwickelnden Anforderungen an eine webbasierte und offline fähige \ac{PLE} zu erfüllen. Eröffnet wird das Kapitel mit einer Vorstellung bestehender \ac{PLE}-Ansätze und Systeme. Es folgt ein Blick auf zwei unterschiedliche Konzepte zum Aufbau und zur Klassifikation von \acp{PLE}, welche bei dem Design und der Evaluierung des Systems von Nutzen sind. Anschließend werden die aktuell verbreitetsten Widget"=Frameworks beschrieben. Es folgen zwei Abschnitte zu dem technologischen Fundament des Systems. Diese Abschnitte sind ausführlicher als normalerweise üblich, da die Rahmenbedingungen der vorliegenden Arbeit die Entwicklung einer \ac{PLE} auf Basis aktueller Technologien vorgeben, so dass ein großer Teil der Arbeit sich mit diesen Thematiken beschäftigt. Das Kapitel schließt mit einem Abschnitt über die Verwendung des Prototypen auf einem USB"=Stick. 

\section{Vergleich mit ähnlichen Systemen}\label{section:aehnliche_systeme}

\subsection{Erweiterungen von \acsp{LMS} zu \acsp{PLE}}
Wie in Kapitel \ref{chapter:Kapitel2} beschrieben gibt es zwei unterschiedliche Arten vom E"=Learning Systemen. \aclp{LMS} (\acp{LMS}) und \aclp{PLE} (\acp{PLE}). Vertreter der klassischen \acp{LMS} sind beispielsweise \href{https://moodle.org/}{Moodle}\footnote{weitere Informationen auf: \url{https://moodle.org/}}, \href{http://www.sakaiproject.org/}{Sakai}\footnote{weitere Informationen auf: \url{http://www.sakaiproject.org/}}, \href{http://www.adrenna.com/}{Adrenna}\footnote{weitere Informationen auf: \url{http://www.adrenna.com/}} oder \href{http://www.blackboard.com/}{Blackboard}\footnote{weitere Informationen auf: \url{http://www.blackboard.com/}}. Diese Systeme verfolgen den sehr kurszentrierten Ansatz klassischer Lern"=Management"=Systeme und widersprechen damit dem \ac{PLE}"=Ansatz. Es gibt, insbesondere für Moodle, Bestrebungen \ac{PLE}"=Funktionalitäten in das System mit aufzunehmen. Die Ansätze hierfür reichen von der Idee, einfach allen Studenten die Rechte zu geben eigene Kurse anzulegen (vgl. Diskussion in \cite{MoodleForum2009}), so dass sie sich einen eigenen Arbeitsplatz zusammenzustellen können, bis hin zu dem Versuch Services aus dem Web 2.0 über Widgets mit dem Moodle"=Kern kommunizieren zu lassen (vgl. \cite{Penalvo2011}). Letzteres mündete in die Entwicklung des \href{http://www.moodbile.org/}{Moodbile}\footnote{weitere Informationen auf: \url{http://www.moodbile.org/}} Systems, welches das Ziel hat mobile Anwendungen mit Hilfe von \ac{W3C}"=Widgets und Wookie in Moodle einzubinden. Dieses Projekt wird jedoch seit einiger Zeit nicht weiterentwickelt und ist nicht kompatibel zu aktuellen Moodle"=Versionen. Nach \cite{Penalvo2011} befindet sich das Ziel \acp{PLE} mit \acp{LMS} zu verbinden und integriert miteinander arbeiten zu lassen noch in weiter Ferne. Systeme wie Blackboard oder Adrenna bauen auf aktuellen Webtechnologien auf und geben trotz ihres institutszentrierten Charakters dem Anwender die Möglichkeit sich mit Services wie Youtube oder Twitter zu verbinden. Beide Systeme müssen jedoch von einem Institut installiert und den Anwendern zu Verfügung gestellt werden, können also nicht von Anwendern außerhalb des Institutes genutzt werden. Sie bieten zwar \acp{API} zur Implementation neuer Funktionalitäten, ihr Quellcode ist jedoch nicht freigeben und unterliegt proprietären Lizenzen.

\subsection{Existierende Ansätze für \acsp{PLE}}
In den Bereich der \acp{PLE} fallen Systeme, die es dem Anwender erlauben sich sein eigenes System aus Widgets zusammenzubauen, also wirklich als Informationsaggregatoren zu fungieren. Die zwei wichtigsten und verbreitetsten Vertreter dieser Art Anwendungen sind das von Google entwickelte \href{http://www.google.de/ig}{iGoogle}\footnote{weitere Informationen auf: \url{http://www.google.de/ig}} und \href{http://www.netvibes.com}{Netvibes}\footnote{weitere Informationen auf: \url{http://www.netvibes.com}}. Beide Systeme verfolgen den Ansatz einer personalisierbaren Homepage. Es ist möglich Widgets für die unterschiedlichsten Services aus einer Widget"=Datenbank auszuwählen und dem System zuzuordnen. Die Widgets können innerhalb des Systems zu unterschiedlichen frei definierbaren Arbeitsflächen, sogenannten Workspaces oder Tabs, zugewiesen und dort per Drag and Drop angeordnet werden. Als Widget"=Format benutzt Netvibes die eigens entwickelte "`Universal Widget API"' (vgl. \cite{UWAJSRuntime2012}), welche erlaubt Widgets auf Basis von \ac{HTML}, \acs{XML} (\aclu{XML}) und Javascript zu entwickeln. Für iGoogle hat Google die sogenannten Google Gadgets (vgl. \cite{GoogleGadgetsApi2012}) entworfen. Die OpenSocial Gadgets, welche in Abschnitt \ref{section:opensocial_gadgets} näher vorgestellt werden, basieren zum Großteil auf der Google Gadgets Spezifikation. Zusätzlich zu den eigenen Widgets werden auf der iGoogle Startseite neben einem Standardsuchfeld für die Google Suche, nicht entfernbare Direktlinks zu anderen Google"=Services wie Calender, Google Drive oder Google+ angeboten. Weder das Netvibes noch das iGoogle Grundsystem stehen als Open Source im Quellcode zur Verfügung. Beide Systeme sind kostenlos, zusätzliche Funktionen (insbesondere für Business Anwendungen) können bei Netvibes für 499 Dollar im Monat dazu gebucht werden. Google hat angekündigt das iGoogle"=System zu Gunsten des Chrome App Shops oder Google Play Marktplatzes für Android"=Systeme zum 1. November 2013 abzuschalten (vgl. \cite{Google2012}).

In ihrer Funktionalität und Arbeitsweise ähneln Netvibes und iGoogle dem in dieser Arbeit zu entwickelnden System. Beide Systeme besitzen jedoch keine expliziten Funktionalitäten zum Offline"=Arbeiten. Netvibes erlaubt es einzelne Artikel zum Beispiel in einem Newsfeed als "`zum später Lesen"' zu markieren und herunterzuladen. Es gibt jedoch weder die Möglichkeit die Seite erneut zu laden, wenn keine Internetverbindung besteht, noch die Möglichkeit offline weiterzuarbeiten und vorgenommene Änderungen bei einer erneuten Verbindung zu synchronisieren. 

Zusammenfassend gibt es momentan auf dem Markt keine Systeme, welche alle in Kapitel \ref{chapter:Kapitel3} definierten Anforderungen erfüllen können. Aus diesem Grund wurde für diese Arbeit ein komplett neues System entworfen und implementiert.

\section{Konzepte zum Aufbau und zur Klassifikation von \acsp{PLE}}
Das Ziel dieser Arbeit ist der Entwurf und die prototypische Implementierung einer leichtgewichtigen \ac{PLE} mit Offline"=Funktionalitäten auf Basis aktueller Technologien. Zum Erreichen dieses Zieles ist es hilfreich sich einen Rahmen zu schaffen, in dem diese Implementierung stattfinden kann. Ein solcher Rahmen kann helfen ein Vokabular zu erstellen, auf dessen Basis die benötigten Funktionalitäten der \ac{PLE} beschrieben und kategorisiert werden können. Zur Schaffung dieses Rahmens wurden für diese Arbeit zwei Konzepte gewählt, welche auf Basis unterschiedlicher Ansätze Funktionalitäten von \acp{PLE} auf unterschiedliche Klassen abbilden. Zum einen gibt es den Versuch von \cite{Palmer2009} Dimensionen zu definieren und die Funktionalitäten von \acp{PLE} diesen Dimensionen zuzuordnen. Palmér schafft damit ein System, um unterschiedliche \ac{PLE}"=Systeme kategorisieren zu können. Zum anderen definiert \cite{Wilson2008} Entwurfsmuster für \acp{PLE}. Diese Muster sollen helfen wiederkehrende Konzepte beim Entwurf einer \ac{PLE} zu beschreiben und Lösungsansätze für häufig auftretende Probleme zu bieten. Diese beiden Konzepte werden im Folgenden vorgestellt. Anschließend wird unter Bezugnahme auf die Anforderungsanalyse begründet auf welchen Teilen der Schwerpunkt bei der Umsetzung dieser Arbeit liegt.

\subsection{Dimensionen nach Palmér}\label{section:dimensions_palmer} 
Palmér definiert sechs Dimensionen mit denen er so viele relevante Funktionalitäten von \acp{PLE} wie möglich erfassen möchte (vgl. \cite{Palmer2009}). Diese Dimensionen sollen relativ unabhängig voneinander sein, so dass es möglich ist, dass unterschiedliche Plattformen einige Dimensionen mehr und andere weniger berücksichtigen und implementieren. Eine \ac{PLE} kann folglich anhand des Grades ihrer Implementierung der einzelnen Dimensionen kategorisiert und bewertet werden.

Die \emph{Screen"=Dimension} befasst sich mit Aspekten, welche die Darstellungsebene von \acp{PLE} definieren. Hierzu zählt Palmér insbesondere das User"=Interface und die Usability des \ac{PLE}"=Containers (wie sind die Widgets angeordnet, wie können neue Widgets gesucht und hinzugefügt werden, wie einfach kann sich der Nutzer im System bewegen etc.), welcher als Einstiegspunkt in die Systembedienung und als User"=Interface für die einzelnen Widgets dient.

Mit der \emph{Data"=Dimension} beschreibt Palmér Funktionalitäten, die für die Portabilität der verwendeten Daten innerhalb einer \ac{PLE} notwendig sind. Idealerweise sollen Widgets in der Lage sein untereinander und mit dem \ac{PLE}"=Container zu kommunizieren. Sie sollen Daten austauschen können und sich so weit wie möglich über ihre Zustände informieren. Des Weiteren soll es möglich sein die Daten der Widgets zu exportieren und sie an anderer Stelle oder in einem anderen \ac{PLE}"=Container wieder zu importieren und weiterzuverwenden. Mit der fortschreitenden Mobilität der Nutzer wird es immer wichtiger, dass der Zugriff auf die \ac{PLE} auch dann möglich ist, wenn kein Zugriff auf das Internet besteht. Somit ist es nicht nur notwendig Daten zu importieren und zu exportieren, sondern auch zwischen einem Offline und einem Online"=Speicher zu synchronisieren, sobald eine entsprechende Zustandsänderung eintritt.

Ein wesentlicher Bestandteil des Web 2.0 ist die Vernetzung von Freunden und Menschen mit ähnlichen Interessen untereinander. Dem trägt Palmér mit der \emph{Social"=Dimension} Rechnung. Diese Dimension gibt an, wie sehr eine \ac{PLE} Funktionalitäten sozialer Netzwerke wie Freundeslisten integriert und Möglichkeiten bietet, den Zugriff auf geteilte Ressourcen auf bestimmte Typen von Freunden einzuschränken.

Die in der Screen"= und der Social"=Dimension beschriebene Kollaboration zwischen Nutzern der \ac{PLE} bringt neue Anforderungen mit sich. So ist es beispielsweise notwendig oder zumindest wünschenswert, dass geänderte Inhalte sich in Echtzeit in den Instanzen der Widgets manifestieren, die ebenfalls auf diese Inhalte zugreifen. Hierbei sollten auch Probleme, wie auftretende Konflikte bei gleichzeitigem Bearbeiten der selben Ressource in Betracht gezogen werden. Diese Anforderungen beschreibt Palmér in der \emph{Temporal"=Dimension}.

Die \emph{Activity"=Dimension} beschreibt die Möglichkeit Abläufe und Workflows innerhalb einer \ac{PLE} aktiv zu gestalten. Hierzu gehören unterer anderem einfache Dinge wie Anleitungen als Hilfestellung für den Nutzer zur Bewegung innerhalb der \ac{PLE}. Besonderen Wert legt Palmér jedoch auf die Abbildung von Lernsequenzen innerhalb der \ac{PLE}. So können bestimmte Widgets auf bestimmte Ereignisse reagieren oder sich selbst aktivieren oder deaktivieren. Des Weiteren können unterschiedlichste Konzepte aus dem Bereich des E"=Learnings, wie zum Beispiel die Learning Design Specification des IMS Global Learning Consortium (vgl. \cite{IMS2012}) oder Learning Object Metadata (LOM) (vgl. \cite{LOM2002}), in den Widgets oder dem \ac{PLE}"=Container implementiert werden.

Die \emph{Runtime"=Dimension} befasst sich mit Funktionalitäten, die die Interoperabilität zwischen \ac{PLE}"=Systemen und Komponenten beinhalten. Nach Palmér werden Nutzer in der Zukunft nicht nur eine \ac{PLE} nutzen, sondern je nach Bedürfnis oder Anwendungsfall zwischen unterschiedlichen Systemen hin- und herwechseln. Hierfür sollte es möglich sein Importe und Exporte für Inhalte und Einstellungen sowohl der einzelnen Widgets als auch des \ac{PLE}"=Containers vorzunehmen.

\subsection{Muster für das Design von \acsp{PLE}}\label{section:wilson_patterns}
\cite{Wilson2008} stellt unterschiedliche Muster vor, die bei der Entwicklung einer \ac{PLE} helfen sollen. Hierbei legt er seinen Fokus auf die dezentralisierte  nutzerzentrierte Natur von \acp{PLE}. Als Motivation dafür gibt er an, dass \acp{PLE} nicht ein einfaches Stück Software seien. Vielmehr stellen sie eine Umgebung dar, in der Menschen mit Werkzeugen, Ressourcen und Communities in einer lockeren und nicht vorher strikt fixierten Art und Weise kommunizieren. Nach Wilson erleichtert diese lockere Aufbau es dem Entwickler jedoch für den Anwender gut nutzbare Systeme zu entwerfen. Wilson leitet seine Muster von denen ab, die von der Universität von Bolton während des "`Personal Learning Environments Reference Model Project"' Projektes entwickelt wurden. Er versucht die Anzahl der Muster jedoch von ursprünglich 77 deutlich zu reduzieren. Hierfür fasst er Muster, die sich auf einfache Funktionen beziehen zu größeren Gruppen zusammen und beschreibt so eher allgemeine Charakteristika von \acp{PLE}.

Wilson unterscheidet zwischen zwei verschiedenen Arten von Mustern für \acp{PLE}: Muster für persönliche Anwendungen (Personal Tools) und Muster für Lernnetzwerke (Learning Networks). Personal Tools stellen hierbei die Werkzeuge dar, die ein Nutzer direkt für seine Lernaktivitäten nutzt. Er interagiert mit unterschiedlichen sozialen Netzen (zum Beispiel Lernnetzwerken) oder verwendet die Tools anderweitig für seine persönliche Art des Lernens. 
Wilson definiert Learning Networks als die Infrastruktur, welche notwendig ist, um Lernnetzwerke mit Hilfe sozialer Netze oder Communities aufzubauen. Des Weiteren zählt er auch die Menge der Online"=Services hinzu, welche von einem Lehrinstitut den Lernenden zur Verfügung gestellt werden (vgl. \cite{Wilson2008}).

Diese Arbeit beschäftigt sich mit dem Aufbau einer \ac{PLE}, welche direkt vom Nutzer für seine Lernaktivitäten verwendet wird. Learning Networks spielen hierfür keine oder eine nur sehr untergeordnete Rolle, so dass die Muster für das Erstellen eben dieser hier nicht weiter beleuchtet werden. Im Folgenden werden die Muster für Personal Tools vorgestellt. Diese ähneln meist einer Empfehlung, wie eine bestimmte Funktion umgesetzt werden sollte oder welche Funktionalitäten vorhanden sein sollten, um die Erfahrung des Nutzers im Umgang mit dem System zu verbessern.

In einer PLE werden Informationen aus potentiell sehr vielen und unterschiedlichen Quellen verarbeitet. Um dem Nutzer die Möglichkeit zu geben wichtige Informationen schneller herauszufiltern oder zu priorisieren, sollte ein \emph{Discourse Monitor} implementiert werden. Dieser fasst die wichtigsten Informationen aus den unterschiedlichen Quellen zusammen und bereitet sie in übersichtlicher Art und Weise auf. Dem Nutzer soll es dann möglich sein, die dargestellten Daten zu Filtern, zu priorisieren, seine Favoriten zu kennzeichnen oder neu eingetroffene Daten einfach und zeitnah zu erkennen. 

Der Nutzer einer PLE ist bei verschiedensten Netzwerken angemeldet und hat dort höchstwahrscheinlich unterschiedliche Informationen hinterlegt. Es ist auch möglich, dass eine Kommunikation über Netzwerkgrenzen hinweg stattgefunden hat. Diese Daten sind meist nicht in einem einzigen Netzwerk darstellbar. Ein \emph{Connection Hub} soll in der Lage sein oder den Nutzer in die Lage versetzen, diese Verbindungen darzustellen und aufzubereiten. Es wäre beispielsweise vorstellbar, dass es dem Nutzer ermöglicht wird Informationen und Daten verschiedenster Netzwerke zu kombinieren,ohne dass diese Netzwerke direkt in Kontakt zueinander stehen.

Es ist unabdingbar, dass während des computergestützten Lernens sowohl von dem Lernenden, als auch von dem Lehrenden Ressourcen erstellt und diese auch anderen zur Verfügung gestellt werden. Hierzu gehören alle Arten von Ressourcen, also Textdateien, Notizen, Quellcode, Videos, Bilder, Präsentationen etc. Die \ac{PLE} sollte es dem Nutzer ermöglichen diese Ressourcen zu erstellen (auch unter Zuhilfenahme externer Services) und diese dann an unterschiedliche Netzwerke zu verteilen, was Wilson mit dem \emph{Create and Mix Media}-Muster ausrdrückt

Durch das Nutzen unterschiedlicher Services hat der Nutzer höchstwahrscheinlich auch Useraccounts bei all diesen Services erstellt. Wilson schlägt in dem \emph{Integrate Identities}-Muster für die Vereinfachung des Umgangs mit diesen Accounts die Implementierung von unterschiedlichen Mechanismen vor. Hierzu gehören die Nutzung von Systemen, die in der Lage sind Zugangsdaten für unterschiedliche Accounts zu verwalten und zu speichern. Es können aber auch Konzepte wie das zentrale Hinterlegen von Profildaten, zum Beispiel bei einem OpenID Server"=Anbieter\footnote{weitere Informationen auf: \url{http://openid.net/}}, verwendet werden. Die Registrierung und der Login bei unterschiedlichen Services erfolgt dann über den Umweg eines zentralen Services, welcher alle notwendigen Daten zu dem Service übermittelt, bei dem sich der Nutzer einloggen möchte.  

Benutzt man ein System wie eine \ac{PLE} zur Organisation des persönlichen Lernens, so sollte einem dieses System Werkzeuge an die Hand geben, um das Lernen und die Arbeit zu organisieren. Hierzu gehören Zeitplaner, Kalender, Todo"=Listen oder auch die Möglichkeit zur Erstellung von Notizen innerhalb des Systems. Diese Werkzeuge werden von Wilson in dem \emph{Manage Time and Effort}-Muster zusammengefasst.

Bei einer \ac{PLE} hat sich der Nutzer idealerweise seine Lernumgebung aus unterschiedlichen Quellen und Werkzeugen zusammengestellt. Der \emph{Navigation Layer} fasst diese in einem System zusammen und ermöglicht dem User einen einfachen Zugriff auf seine Werkzeuge. Wilson schlägt vor die Services als Widgets in der \ac{PLE} einzubinden. Somit wird die \ac{PLE} zu einem zentralen Zugriffspunkt oder zu einem Dashboard von dem aus der Nutzer alle seinen Aktionen ausführen kann. 

Die Nutzergewohnheiten bezüglich des Gebrauchs des Internets haben sich in den letzten Jahren deutlich gewandelt (vgl. \cite{VanHarmelen}). Der Zugriff erfolgt nicht mehr primär über den eigenen (Heim-)Rechner, sondern über die verschiedensten Zugriffspunkte. Hierzu gehören Computer in der Universität, am Arbeitsplatz, im Internetcafé und in letzter Zeit verstärkt auch mobile Geräte wie Smartphones oder Tablets. Aus diesem Grunde ist es notwendig, dass der Zugriff auf Lernnetzwerke von all diesen Geräten aus möglich ist und die Systeme auch gut von den unterschiedlichen Geräten aus bedienbar sind. Des Weiteren sollte es ein System geben, welches die Daten auf den unterschiedlichen Geräten miteinander synchronisiert, so dass der Nutzer von überall auf die aktuellsten Informationen zugreifen kann. Diese Funktionalitäten werden durch das \emph{Multi-platform/Multimode}-Muster beschrieben.

Das \emph{Choose, Change, Discard}\label{wilson_patterns:choose_change_discard} Muster steht in engem Bezug zu der nutzerzentrierten Herangehensweise in dem Aufbau von \acp{PLE}. User sollen in der Lage sein, sich ihre Lernumgebung nach eigenen Vorlieben und Anforderungen einzurichten. Außerdem ist es sehr gut möglich, dass sich die Anforderungen im Laufe der Zeit ändern. Aus diesen Gründen ist es notwendig, dass es dem Nutzer frei steht Inhalte und Services innerhalb der \ac{PLE} zu verschieben und anzupassen und neue Werkzeuge hinzuzufügen und nicht mehr benötigte Werkzeuge wieder zu entfernen. Die \ac{PLE} sollte hierbei nicht zu viele vom Nutzer nicht änderbare Bedienungsvorgaben und Konfigurationseinstellungen machen, sondern dem Anwender so viele Freiheiten wie möglich bei Einrichten der eigenen Lernumgebung zu geben.

\subsection{Einordnung und Eingrenzung der Funktionalitäten der zu entwickelnden \acs{PLE}}
Dieser Abschnitt nutzt die Muster nach Wilson und die Dimensionen nach Palmér, um die zu entwickelnde \ac{PLE} in die gegebenen Kategorien einzuordnen.
Die funktionalen Anforderungen (siehe Tabelle \ref{table:funktionale_anforderungen}) erfordern, dass ein User"=Interface des \ac{PLE}"=Containers erstellt wird. Zusätzlich verlangt die nichtfunktionale Anforderung (\reqref{requirementUsageInBrowser}), dass dieses Interface so umgesetzt werden muss, dass es einem aktuellen Browser ohne weiteren Installationsaufwand lauffähig ist. Der erste Teil des Entwurfes beschäftigt sich also mit der Umsetzung der \emph{Screen-Dimension} nach Palmér. Zur Umsetzung der Anforderungen im User"=Interfaces eignen sich insbesondere die von Wilson definierten \emph{Discourse Monitor}, \emph{Navigation Layer} und \emph{Choose Change and Discard} Muster. Als Einstiegspunkt in das System kann ein \emph{Discourse Monitor} dienen. Dieser fungiert als eine Startseite welche dem Nutzer die wichtigsten Informationen bezüglich seiner Widgets zur Verfügung stellt  und ihm durch Links die Möglichkeit gibt direkt zu den gewünschten Personal"=Learning"=Tools zu gelangen. Somit kann das System als ein Aggregator für unterschiedliche externe Kanäle und Services dienen (\reqref{requirementAggregator}). Der \emph{Navigation Layer} muss in einer \ac{PLE}, welche als Mashup"=Anwendung konzipiert wird, inhärent mit eingebaut werden. Durch die Einbindung externer Widgets wird dem Anwender nur der Funktionsumfang zur Verfügung gestellt, welche von den Widget"=Entwicklern angedacht wurde. Für den vollen Funktionsumfang muss der Nutzer zum eigentlichen Service des Widgets navigieren. Die Widgets sollten dann innerhalb des Systems aggregiert und ihm in übersichtlicher Form in Form eines Dashboards präsentiert werden (\reqref{requirementDashboard}). Die Anforderungen \reqref{requirementWorkspaceAdd}, \reqref{requirementWorkspaceEdit}, \reqref{requirementWorkspaceDelete}, \reqref{requirementWidgetAdd}, \reqref{requirementWidgetDelete} und \reqref{requirementWidgetSortDragNDrop} erfordern, dass der Anwender Workspace und Widgets hinzufügen, bearbeiten, entfernen und in ihrer Position verändern können soll. Somit wird auch das \emph{Choose Change and Discard} Pattern grundsätzlich im System verankert. Die Anforderungen \reqref{requirementOfflineWork}, \reqref{requirementOnlineSync} und \reqref{requirementOfflineStart} verlangen die Fähigkeit die wichtigsten Funktionalitäten auch offline weiterhin nutzen zu können. Wenn die Konnektivität wieder hergestellt ist, sollen die Daten synchronisiert werden. Die Umsetzung dieser Anforderung beschäftigt sich also primär mit den Umgang mit Daten und Informationen und den Austausch ebendieser zwischen unterschiedlichen Systemen (zwischen dem System und den Widgets) und der Synchronisierung der Daten bei Statusänderung (von Offline- zu Online"=Modus) und fallen folglich in Palmérs \emph{Data-Dimension}. Die benötigte Fähigkeit des Systems zu erkennen, ob es zu einem Zeitpunkt online oder offline ist (\reqref{requirementCheckOnlineStatus}) kann als eine Erweiterung des von Wilson vorgestellten \emph{Multimode-Patterns} betrachtet werden.   
Die restlichen Dimensionen und Muster werden im Zuge dieser Arbeit nicht implementiert, da dies über den Rahmen dieser Arbeit weit hinausgehen würde und nicht nötig ist, um die definierten Anforderungen zu erfüllen.

\section{Widget-Frameworks}\label{section:widget_frameworks}
Damit die \ac{PLE} als Aggregator für unterschiedliche Services und Kanäle verwendet werden kann, müssen diese Services als Widgets (siehe Kapitel \ref{section:widgets}) eingebunden werden. \reqref{requirementWidgetStandard} fordert, dass hierfür ein frei verfügbarer Standard genutzt wird. Um dies zu erfüllen wurden im Zuge dieser Arbeit die beiden wichtigsten frei verfügbaren Widget"=Frameworks untersucht und miteinander verglichen. Im Folgenden werden diese zwei Frameworks vorgestellt. Anschließend wird eine Begründung für die Wahl eines der Framework für diese Arbeit gegeben.

\subsection{\ac{W3C}-Widgets}\label{section:w3c_widgets}
Die in einer \ac{W3C}"=Empfehlung spezifizierten Widgets sind komplette \ac{HTML}/Javascript Anwendungen, dessen gesamte Ordnerstruktur zu einer gepackten Zip"=Datei zusammengefasst wird (vgl. \cite{W3C-11-2012}). Die Spezifikationen verlangen, dass in dieser Datei neben der Anwendung selbst noch eine \ac{XML}"=basierte Konfigurationsdatei vorhanden ist, welche innerhalb eines Widget"=Wurzelelementes die Eigenschaften des Widgets definiert. Zu den Eigenschaften zählt die Widget"=Größe (in Pixeln), der \ac{XML}"=Namensraum, Icons, die in Widget"=Vorschauen benutzt werden sollen, externe Features, welche das Widget benötigt, aber auch Informationen über den Autor oder die Lizenz unter der das Widget steht. Über das Element $<$content$>$ wird innerhalb der \ac{XML}-Datei wird der eigentliche Inhalt des Widgets geladen. Dieser kann also unabhängig von der späteren Verwendung innerhalb eines Widgets implementiert werden.

Neben der erwähnten Recommendation gibt es noch andere Spezifikationen rund um die Widget"=Thematik, welche unterschiedliche Teilbereiche wie zum Beispiel die digitale Signierung von Widgets standardisieren\footnote{weitere Informationen auf: \url{http://www.w3.org/2008/webapps/wiki/WidgetSpecs}}. Als Referenzimplementierung für die W3C"=Widgets fungiert das Apache"=Projekt "`Wookie"'\footnote{weitere Informationen auf: \url{http://wookie.apache.org/}}. Wookie ist ein in Java"=implementierter Widget"=Container, welcher die wichtigsten Funktionalitäten für die Arbeit mit \ac{W3C}"=Widgets zur Verfügung stellt. Widgets, die nach dem \ac{W3C}"=Standard erstellt wurden, können in den Wookie Container geladen werden. Dieser übernimmt dann die Auslieferung inklusive dem benötigten Javascript"=Code an die anfordernden Anwendungen. Wookie benutzt intern eine Datenbank, in der Widget"=Einstellungen für jeden User oder für jede ausgelieferte Instanz eines Widgets gespeichert werden können. Mit der in den \ac{W3C}"=Standards spezifizierten \ac{API} können diese Einstellungen über Javascript geändert oder ausgelesen werden. Die Administration von Wookie erfolgt entweder über die Kommandozeile oder über eine REST"=\ac{API}{weitere Informationen auf: \url{http://wookie.apache.org/docs/admin.html}} (siehe Abschnitt \ref{section:rest}). Für die Einbindung in andere Applikationen bietet Wookie eine Connector"=\ac{API} an, für welche Beispielimplementationen für Java und für PHP existieren\footnote{weitere Informationen auf: \url{http://wookie.apache.org/docs/embedding.html}}.

\subsection{OpenSocial Gadgets}\label{section:opensocial_gadgets}
Nach der \cite{Opensocial2013} ist OpenSocial
\begin{quotation}
[...] a set of APIs for building social applications that run on the web. OpenSocial's goal is to make more apps available to more users, by providing a common API that can be used in many different contexts. Developers can create applications, using standard JavaScript and HTML, that run on social websites that have implemented the OpenSocial APIs. These websites, known as OpenSocial containers, allow developers to access their social information; in return they receive a large suite of applications for their users (\cite{Opensocial2013}).
\end{quotation}
OpenSocial hat sich also zum Ziel gesetzt, eine allgemeine \ac{API} für die Erstellung von Social Software zu entwickeln. Hierzu definiert es allgemein Konzepte wie "`Freundeslisten"', "`Aktivitäten"' und "`Profile"', standardisiert aber auch Authentifizierungs- und Authorisierungsmechanismen. Ein wichtiger Bestandteil der OpenSocial"=\ac{API} sind die OpenSocial Gadgets. Diese basieren auf den von Google entwickelten "`Google Gadgets"' (vgl. \cite{GoogleGadgetsApi2012}) und stellen ähnlich wie die \ac{W3C}"=Widgets eine API bereit, um eigenständige Webapplikationen in Mashup Anwendungen zu integrieren. Ähnlich wie für die \ac{W3C}"=Widgets benötigen OpenSocial Gadgets eine \ac{XML}"=Konfigurationsdatei. Der große Unterschied zu den \ac{W3C}"=Widgets liegt jedoch darin, dass sämtlicher \ac{HTML}- und Javascript"=Quellcode des Widgets ebenfalls in dieser \ac{XML}"=Datei hinterlegt wird. Der komplette Inhalt befindet sich also nicht in einer eigenen \ac{HTML}"=Datei, sondern in einem Knoten innerhalb des \ac{XML}"=Dokumentes (vgl. \cite{GoogleGadgetsApi2012}). Eine Referenzimplementierung für die OpenSocial Gadgets ist der Apache "`Shindig"' Server\footnote{weitere Informationen auf: \url{http://wookie.apache.org/}}. Dieser fungiert ähnlich wie Apache Wookie als Gadget"=Container und kann gerenderte Widgets an Applikationen ausliefern. Es existieren Shindig"=Implementierungen in Java und in PHP.

\subsection{Wahl des Frameworks}
Für die Umsetzung in dieser Arbeit fiel die Wahl auf das Framework des \ac{W3C} und den Widget"=Container Wookie. Somit wird der zu erstellende Prototyp in der Lage sein alle Widgets, die dem \ac{W3C}"=Standard entsprechen, einzubinden. Die Wahl hätte auch auf die OpenSocial Gadgets und Shindig als Widget"=Container fallen können. Den Ausschlag für die \ac{W3C}"=Widgets und Wookie gaben unterschiedliche, vor allem praktikable Gründe. Das wichtigste Argument für die Nutzung von \ac{W3C}"=Widgets ist, dass es möglich ist sie als ganz normale Web"=Anwendungen zu entwickeln und zu testen. Es ist lediglich notwendig eine gesonderte \ac{XML}"=Konfigurationsdatei mit einigen Parametern zu hinterlegen. Wookie stellt dann Funktionalitäten zur Verfügung, die es erlauben aus einem einfachen Verzeichnis im Dateisystem ein nach \ac{W3C}"=Spezifikationen gepacktes Widget zu erstellen und dieses auch direkt in Wookie bereitzustellen. Für OpenSocial Gadgets muss der gesamte \ac{HTML}"= und Javascript"=Quellcode innerhalb der \ac{XML}"=Datei liegen. Zwar gibt es auch hier Werkzeuge zur Vereinfachung der Gadget"=Erstellung, aber der \ac{W3C}/Wookie"=Weg erschien praktikabler. Des Weiteren war es zum Zeitpunkt der Entscheidungsfindung um die Dokumentation für die OpenSocial"=Seite insbesondere von Shindig eher schlecht bestellt. Viele Verweise auf (z.B. auf Installationsanleitungen) führten ins Leere und die PHP"=Referenzimplementation war vom Entwicklungsstand ca. 1,5 Jahre hinter dem Java System zurück und hatte viele nicht behobene Fehler. Dieser Zustand scheint sich jedoch zum aktuellen Zeitpunkt geändert und verbessert zu haben. Die Dokumentation von den \ac{W3C}"=Widgets und Wookie waren und sind relativ aktuell und es existiert eine funktionierende Implementation der Connector"=\ac{API} in PHP. Diese kann als Grundlage für die zu implementierende Kommunikation des Basissystems mit dem Widget"=Container verwendet werden.

\section{Technologien für Offline-Web-Anwendungen}
Die Anforderungen \reqref{requirementOfflineWork}, \reqref{requirementOnlineSync}, \reqref{requirementOfflineStart} und \reqref{requirementCheckOnlineStatus} befassen sich mit den Offline"=Fähigkeiten des zu entwickelnden Systems. Wenn alle diese Anforderungen erfüllt sind, kann man von einer "`Offline"=Web"=Anwendung"' sprechen. Dieser Begriff mag sich zunächst nach einem Widerspruch anhören. Normalerweise kennt der Nutzer den folgenden Arbeitsablauf, wenn er sich im Internet bewegt: Öffnen des Internetbrowser $\Rightarrow$ eingeben der gewünschten \ac{URL} $\Rightarrow$ der Browser verbindet sich als Client mit dem hinter der \ac{URL} stehenden Server und lädt die zur Verfügung gestellten Inhalte (\ac{HTML}, Javascript, \ac{CSS}, Medienressourcen etc.) herunter $\Rightarrow$ der Browser rendert das \ac{HTML}/\ac{CSS} und führt die Anweisungen in dem Javascript"=Quellcode aus, so dass dem Nutzer eine funktionsfähige Seite zur Verfügung steht. Ist das System nun aber offline, kann der Browser keine Verbindung mit dem Server herstellen und erhält somit keine Daten, die er darstellen kann. Es gibt Systeme auf \ac{HTML}/Javascript Basis, die niemals den Kontakt mit dem Internet benötigen (vgl. \cite{Mahemoff22010}). Diese Anwendungen werden komplett aus dem Dateisystem geladen und benutzen den Browser nur für die Darstellung (z.B. \href{http://tiddlywiki.com}{TiddlyWiki}\footnote{weitere Informationen auf: \url{http://tiddlywiki.com/}}, ein Wiki"=System, welches komplett lokal auf dem Rechner des Anwenders läuft). Diese Systeme sind aber für diese Arbeit irrelevant, da die erwähnten Anforderungen zeigen, dass für das zu entwickelnde System Anwendungen notwendig sind, die je nach Notwendigkeit sowohl online als auch offline arbeiten können. All diesen Systemen gemein ist, dass sie zumindest eine initiale Internetverbindung benötigen, bei der Zugriff auf die genutzten Ressourcen besteht.

In \ac{HTML}5 gibt es im Wesentlichen zwei Mechanismen, die für die Offline"=Fähigkeiten von Webanwendungen verantwortlich sind: Application"=Caching und Offline"=Storage (vgl. \cite{Mahemoff22010}). Der Unterschied zwischen ihnen liegt in der Art der Daten, die sie lokal vorhalten. Application"=Caching ist für das Caching, also die Speicherung der zentralen Applikationslogik und des User"=Interfaces im Browser zuständig. Dies bedeutet das alle benötigen Ressourcen (\ac{HTML}, Javascript, Iss etc.) heruntergeladen und im Browsercache vorgehalten werden. Dieser Cache bleibt auch bestehen, wenn der Browser geschlossen wird. Letzteres gilt ebenso für den Offline"=Storage. Offline"=Storage ist ein Mechanismus, welche in der Lage ist, die Daten, die der Nutzer eingibt oder bearbeitet ebenfalls im Browser zu speichern und wieder zur Verfügung zu stellen. Gemeinsam mit einem Mechanismus, der es Browsern erlaubt zu erkennen, ob sie online oder offline sind, versetzen Application"=Caching und Offline"=Storage den Entwickler also in die Lage ein System zu entwickeln, welches in der Lage ist ohne Konnektivität mit dem Internet erneut aufgerufen zu werden. Der Anwender kann mit dem System anschließend arbeiten, als wenn eine Internetverbindung zur Verfügung steht.

Im Folgenden werden die Möglichkeiten, die \ac{HTML}5 für die Umsetzung des Application"=Cachings und des Offline"=Storages bietet, genauer beschrieben.

\subsection{Application-Caching}\label{section:appcache}
Die Basis des Application"=Cachings ist eine Manifest Datei (vgl. \cite{Bidelman2010}). In dieser Datei wird hinterlegt, welche Ressourcen vom Webbrowser zwischengespeichert werden sollen. Die Datei ist eine einfache Textdatei, dessen Dateiendung frei gewählt werden kann. Es wird aber empfohlen die Manifestdatei auf .appcache enden zu lassen (vgl. \cite{W3C2012}). Wichtig ist, dass die Datei vom Webserver mit dem MIME type \texttt{text/cache-manifest} ausgeliefert wird.

Jede \ac{HTML}"=Seite die den Application"=Cache nutzen soll, muss diese Datei über das \texttt{ma\allowbreak ni\allowbreak fest}"=Attribut in ihrem html"=Tag einbinden (vgl. \cite{html5upandrunning}). Es ist dabei egal, ob die Datei über einen relativen oder einen absoluten Pfad eingebunden wird. Sie muss nur von der \ac{HTML}"=Seite aus erreichbar sein:
\begin{lstlisting}
<!DOCTYPE html>
<html manifest="cache.appcache">
\end{lstlisting}

Das folgende Listing zeigt den beispielhaften Aufbau einer Manifestdatei und stammt bis auf den Fallback Bereich aus dem entwickelten System:
\begin{lstlisting}
CACHE MANIFEST
#Rev 9
CACHE:
#CSS
/css/compiled/plesynd/main.css
...
NETWORK:
*
FALLBACK:
/ /offline.html
\end{lstlisting}

\cite{W3C2012}, \cite{html5upandrunning} und \cite{Bidelman2010} beschreiben den Aufbau der Datei folgendermaßen: Die Datei beginnt mit einem \texttt{CACHE MANIFEST}"=Header, der angibt, dass es sich um eine Manifestdatei handelt. Es folgt eine Kommentarzeile mit einer Revisionsnummer. Diese Zeile ist nach der Spezifikation eigentlich nicht notwendig. Der Grund für ihr Vorhandensein ist, dass ein Browser den Inhalt des Caches nur aktualisiert, wenn sich die Manifestdatei selbst ändert, nicht aber wenn sich die Dateien, die im Manifest referenziert werden, aktualisiert wurden. Ändert man nun beispielsweise eine Javascript"=Datei, so würde der Browser diese Änderung nicht registrieren. Man muss zusätzlich die Manifestdatei (und sei es nur in einem Zeichen, wie hier die Revisionsnummer) ändern. Der Browser bemerkt dies und aktualisiert sowohl die Datei als auch die von ihr referenzierten Inhalte. Die Zeilen nach dem \texttt{CACHE}"=Eintrag geben die Ressourcen an, die explizit vom Browser gecached werden sollen. Wie man sieht können hier alle möglichen Ressourcentypen, also auch Binärdateien wie Bilder angegeben werden. Die Zeilen die mit einem Hash anfangen, sind nur Kommentare zur besseren Lesbarkeit. Der \texttt{NETWORK}"=Eintrag definiert eine Whitelist, welche angibt welche Ressourcen in jedem Fall über das Netz geholt werden, (auch wenn der Browser offline ist). Hier könnte beispielsweise der Zugriff auf vom Server dynamisch interpretierte PHP oder CGI Dateien definiert werden. Der in diesem Fall verwendete Platzhalter gibt an, dass alle Dateien, die nicht in der \texttt{CACHE}"=Sektion definiert wurden über das Netz geholt werden. Der \texttt{FALLBACK}"=Bereich beschreibt eine alternative Datei, welche angezeigt wird, wenn die angeforderte Ressource nicht im Cache liegt und kein Internetzugang besteht. Im vorliegenden Fall trifft \texttt{/} auf jede Seite ab dem Wurzelverzeichnis zu. Wenn die angeforderte Seite also nicht im Cache liegt, wird die Fallback"=Datei angezeigt. Abschließend ist zu bemerken, das jede \ac{HTML}"=Datei, die auf eine Manifestdatei verweist implizit ebenfalls gecached wird. Der Cache kann also bei einer Webseite, durch die sich der Anwender bewegt sukzessive wachsen.

Mit dem Application"=Cache ist es also möglich alle Ressourcen, die für die Anwendung benötigt werden offline im Browser zwischenzuspeichern. Die Anwendung kann nach einem initialen Download der Ressourcen auch geöffnet werden, wenn der Browser keine Internetverbindung hat. Somit eignet sich diese Technologie ideal für eine Umsetzung der Anforderung \reqref{requirementOfflineStart}.

\subsection{Offline-Storage}\label{section:offline_storage}
In der Vergangenheit wurden vor allem Cookies benutzt, um die Daten eines Anwenders innerhalb des Browsers vorzuhalten (vgl. \cite{Mahemoff22010}). Cookies haben jedoch einige schwerwiegende Nachteile, die es verhindern, dass sie für ein wirkliches Speichern von Daten und Einstellungen des Nutzers Verwendung finden können. Zum einen werden sie bei jedem Request an den Server gesendet, wodurch die Netzwerkverbindung verlangsamt wird und potentiell unverschlüsselte Daten über das Netz geschickt werden (außer bei Benutzung von \ac{SSL}). Zum anderen sind sie auf eine Größe von 4 Kilobyte beschränkt, wodurch es nicht möglich ist eine nennenswerte Datenmenge zu hinterlegen (vgl. \cite{html5upandrunning}). Im Zuge der Entwicklung von \ac{HTML}5 und immer größeren Applikationen, welche primär auf der Client"=Seite ausgeführt werden, wurden mehrere Technologien und \acp{API} entwickelt, die für das Speichern der Daten im Browser verwendet werden können. Im Folgenden werden die wichtigsten drei hiervon, Web"=Storage, Web \ac{SQL} und IndexedDb, vorgestellt. Für alle drei Technologien befolgen Browser eine Same"=Origin"=Policy (für eine genauere Erklärung siehe Kapitel \ref{section:same_origin_policy}), d. h. nur Skripte der selben Quelle haben Zugriff auf den Speicher (vgl. \cite{Mahemoff2010}).

\subsubsection*{Web-Storage}
Web"=Storage basiert auf benannten Schlüssel"=/Wert Paaren, welche mit Hilfe von Javascript im Browser hinterlegt werden können (vgl. \cite{html5upandrunning}). Die \ac{API} hierfür wurde vom \ac{W3C} festgelegt. Demnach können Daten über \texttt{setItem(key, value)} gesetzt, über \texttt{getItem(key)} wieder ausgelesen und über \texttt{removeItem(key)} aus dem Web"=Storage entfernt werden (vgl. \cite{W3C2011}). Wichtig ist jedoch, dass der Local"=Storage nur mit Zeichenketten arbeiten kann (vgl. \cite{W3C2011}). Möchte man numerische Daten wieder auslesen, müssen die mit \texttt{parseInt()} oder \texttt{parseFloat()} umgewandelt werden werden. Javascript"=Objekte müssen in das \ac{JSON} Format umgewandelt werden, damit sie im Web"=Storage hinterlegt werden können. Dies funktioniert am besten über die \texttt{JSON.stringify} und \texttt{JSON.parse}"=Methoden. Es existieren zwei unterschiedliche Implementierungen des Web"=Storage Interfaces: Session"=Storage und Local"=Storage. Während der Session"=Storage nach Schließen des Browsers geleert wird, bleiben die Werte die im Local"=Storage hinterlegt werden auch nach dem Schließen des Browsers bestehen. Die \texttt{localStorage}- und \texttt{sessionStorage}"=Objekte zum Zugriff auf den Local"=Storage und Session"=Storage sind in Javascript global, so dass ein Schreiben und Lesen folgendermaßen aussieht:
\begin{lstlisting}
// schreiben in den Session-Storage
sessionStorage.setItem('key', 'value');
// lesen aus dem Local-Storage
localStorage.getItem('key');
\end{lstlisting}
Pro Origin (siehe Abschnitt \ref{section:same_origin_policy}) erhält der Browser Zugriff auf zwischen 2,5 Megabyte und 10 Megabyte Speicherplatz.

Zusammenfassend ist zu sagen, dass die Vorteile des Web"=Storage sind, dass er in allen aktuellen Browsern verfügbar ist und dass der Zugriff über eine sehr einfache \ac{API} über Schlüssel"=/Wertpaare möglich ist. Der Nachteil des Web"=Storage begründet sich aber ebenfalls in dieser einfachen Speicherung. Alle Werte müssen in Zeichenketten umgewandelt werden. Es ist nicht möglich native Javascript"=Objekte zu hinterlegen und abzufragen. Des Weiteren ist keine Indizierung des Datenbestandes möglich und die Zugriffe auf den Speicher laufen nicht in Transaktionen, so dass im schlimmsten Fall Race"=Conditions\footnote{Definition aus Wikipedia: \quote{Ein kritischer Wettlauf, auch Wettlaufsituation (englisch race condition oder race hazard) ist in der Programmierung eine Konstellation, in der das Ergebnis einer Operation vom zeitlichen Verhalten bestimmter Einzeloperationen abhängt.

Unbeabsichtigte Wettlaufsituationen sind ein häufiger Grund für schwer auffindbare Programmfehler; bezeichnend für solche Situationen ist nämlich, dass bereits die veränderten Bedingungen zum Programmtest zu einem völligen Verschwinden der Symptome führen können(\cite{WikipediaRaceConditions}).}} auftreten können. Aus diesem Grund eignet sich der Web"=Storage nicht für Anwendungen, die auf ein schnelles Durchlaufen eines großen Datenbestandes und ein einfaches Auffinden einzelner Datensätze angewiesen sind.

\subsubsection*{Web \acs{SQL}}
Web \acs{SQL} (\aclu{SQL}) ist oder besser war der Versuch ein relationales Datenbanksystem im Browser zu implementieren. Es ermöglicht alle Abfragen, die aus anderen relationalen Systemen bekannt sind, also Transaktionen, Joins, Counts etc. Ein Beispiel für die Syntax ist folgend aufgeführt (vgl. \cite{W3C2010}):
\begin{lstlisting}
function showDocCount(db, span) {
  db.readTransaction(function (t) {
    t.executeSql('SELECT COUNT(*) AS c FROM docids', [], 
      function (t, result) { /* success callback*/ }, 
      function (t, error) { /* error callback*/ }
\end{lstlisting}
Es wird also die Methode \texttt{executeSql} auf ein Transaktionsobjekt ausgeführt. Diese Methode ermöglicht das Setzen eines Callbacks für den Erfolgs- und eines für den Fehlerfall. 

Alle vorhanden Implementierungen für Web \ac{SQL} basieren auf SQLite\footnote{weitere Informationen auf: \url{http://www.sqlite.org/}}, welches eigene Teilmenge des SQL-92-Standards (vgl. \cite{SQL92}) implementiert. Nach Ansicht des \ac{W3C}s und einiger Browserhersteller (insbesondere Mozilla) sollte ein Web"=Standard nicht auf einer fertigen Technologie, sondern auf einer allgemeinen Spezifikation beruhen (vgl. \cite{W3C2010} und \cite{Ranganathan2010}). Da es aber in dieser Hinsicht keinen Fortschritt in der Entwicklung gab, hat das \ac{W3C} folgenden Eintrag auf der Spezifikationsseite von Web \ac{SQL} hinterlegt:
\begin{quotation}
 This document was on the W3C Recommendation track but specification work has stopped. The specification reached an impasse: all interested implementors have used the same SQL backend (Sqlite), but we need multiple independent implementations to proceed along a standardisation path (\cite{W3C2010}).
\end{quotation} 

\subsubsection*{IndexedDb}
IndexedDb ist ein Versuch einen Standard für ein nichtrelationales Datenbanksystem zum Offline"=Speichern von Daten zu entwickeln. Genau so wie Web"=Storage speichert IndexedDb Schlüssel-/Wertpaare (vgl. \cite{Mehta2012}). Die Unterschiede zum Web"=Storage sind jedoch die, dass als Werte native Javascript Objekte in beliebiger Komplexität hinterlegt werden können und dass es möglich ist den Speicher nach bestimmten Attributen innerhalb der Objekte zu indizieren. Des Weiteren arbeiten alle Aktionen auf der Datenbank innerhalb von Transaktionen (vgl. \cite{MDN2011}).  Jede Aktion läuft in einem "`Request"', so dass es möglich ist ähnlich wie bei Web \ac{SQL} \texttt{success} und \texttt{error} Callbacks zu definieren. Möchte man die Daten wieder auslesen, ist es möglich einen Cursor zu erstellen und alle Daten sequentiell zu durchlaufen. Die Indizes erlauben einem aber auch den direkten Zugriff auf bestimmte Daten, wenn nötig auch über einen Cursor.

\subsubsection*{Zusammenfassung / Wahl der Technologie}
Nach \cite{html5upandrunning} wird bisher nur der Web"=Storage flächendeckend und insbesondere auch in mobilen Browsern unterstützt. Vor allem Mozilla hat sich sehr stark gegen Web \ac{SQL} und für die Verwendung von IndexedDb ausgesprochen und unterstützt auch nur noch die letztere Technologie. Gegen diese Entscheidung gibt es aufgrund der im Gegensatz zu Web \ac{SQL} bedeutend komplexeren \ac{API} und der komplizierten Systementwicklung auf Basis von IndexedDb eine Menge Widerspruch (vgl. \cite{Ranganathan2010}). Momentan sieht es aber danach aus, dass IndexedDb zu einer \ac{W3C}"=Empfehlung wird, während Web \ac{SQL} nur von einzelnen Browserherstellern unterstützt wird.

Anforderung \reqref{requirementOfflineWork} fordert, dass der Anwender zumindest rudimentär mit dem System arbeiten kann, auch wenn keine Verbindung zu dem Backend besteht. Hierfür ist es notwendig, dass vom Anwender erstellte Daten und Änderungen im Browser vorgehalten werden, damit sie bei einer Wiederherstellung der Konnektivität synchronisiert werden können. Für die Umsetzung dieser Anforderung wurde in dieser Arbeit eine Entscheidung für den Web"=Storage und gegen Web"=\ac{SQL} und die IndexedDb"=\ac{API} getroffen. Ein Grund dafür ist, dass der Web"=Storage momentan die größte Verbreitung in aktuellen und insbesondere mobilen Browsern genießt. Insbesondere für Web"=\ac{SQL} ist hier in Zukunft auch keine Verbesserung zu erwarten. Des Weiteren ist die Erstellung einer Applikation auf IndexedDb"= und Web"=\ac{SQL}"=Basis deutlich komplexer als auf Basis von Web"=Storage, welcher für das einfache Hinterlegen von zu synchronisierenden Daten ausreicht. Durch seine einfachen Schlüssel"=/Wertpaare ist er in der Umsetzung und Pflege bedeutet einfacher zu Handhaben als die anderen beiden Systeme. In Bezug auf IndexedDb ist es außerdem problematisch, dass sich die \ac{API} noch nicht in einem finalisiertem Zustand befindet und noch größere Änderungen mit neuen Browserversionen zu erwarten sind.

\subsection{Online-/Offline-Erkennung}\label{section:online_offline_erkennung}
Anforderung \reqref{requirementCheckOnlineStatus} fordert, dass das System seinen Onlinestatus kennen und bemerken muss, wenn sich dieser ändert. In aktuellen Browsern existiert eine \ac{API}, die es dem System erlaubt zu erkennen, ob eine Verbindung mit dem Internet besteht oder nicht. Zum einen gibt es die Eigenschaft \texttt{onLine} des globalen \texttt{navigator}"=Objektes. \texttt{navigator.onLine} is \texttt{true}, wenn eine Netzwerkverbindung besteht und ansonsten \texttt{false}. Zum anderen wird bei Änderung des Online"=/Offline"=Zustandes ein Event ausgelöst, für welches im Code Listener definiert werden können:
\begin{lstlisting}
document.body.addEventListener("online", function () {...} 
document.body.addEventListener("offline", function () {...}.
\end{lstlisting}          
Diese in den Listenern definierten Callbacks werden bei einem Wechsel von online zu offline oder umgekehrt ausgeführt. Es ist jedoch zu beachten, dass unterschiedliche Browserhersteller "`offline"' unterschiedlich definieren. Aktuell wechselt zum Beispiel Chrome bei Verlust der Internetverbindung in den Offline"=Modus, im Firefox muss dieser Modus explizit vom User angeschaltet werden (vgl. \cite{MozBug2011}).

\section{Kommunikation in Mashup-Anwendungen}\label{section:kommunikation_in_mashup_anwendungen}
Die Anforderungen \reqref{requirementAggregator} und \reqref{requirementUsageInBrowser} verlangen, dass das System einen Aggregator, also eine Mashup"=Anwendung (siehe Abschnitt \ref{section:mashup_anwendungen}) darstellt, der unterschiedliche Systeme in aktuellen Browsern zusammenfasst. Des Weiteren soll der Nutzer in einer Art Dashboard Informationen über den Zustand der einzelnen Widgets erhalten (\reqref{requirementDashboard}). Dies erfordert, dass die Widgets in der Lage sind mit der Hauptapplikation zu kommunizieren. Die Widgets werden, wie in Abschnitt \ref{section:widget_frameworks} beschrieben, von einer Wookie"=Instanz, also von einem anderen Server als die Hauptapplikation ausgeliefert. Zusammengenommen zieht dies Probleme mit den Sicherheitsrichtlinien innerhalb der Browser nach sich. Dieses Kapitel behandelt diese sogenannte Same"=Origin"=Policy und stellt Lösungswege vor, wie trotz der gegebenen Einschränkungen eine Mashup"=Anwendung funktionieren kann.

In einer Mashup"=Anwendung müssen mehrere System miteinander kommunizieren. Auch und insbesondere im Hinblick auf eine einfache Erweiterbarkeit (Anforderung \reqref{requirementExtensibility}) ist es von Vorteil, wenn die Bestandteile der Anwendung für ihre Anfragen einen ähnliches Prinzip verwenden. Bei der Verwendung eines gemeinsamen Interfaces würde sich die Interoperabilität zwischen den Teilanwendungen erhöhen Entwicklern das Hinzufügen neuer Anwendungen zu der Anwendung beträchtlich erleichtern. In den letzten Jahren ist hierfür das REST"=Prinzip immer weiter in den Fokus gerückt, welches ebenfalls in diesem Kapitel vorgestellt wird.

\subsection{Same-Origin-Policy}\label{section:same_origin_policy}
Moderne Browser benutzen als Teil ihres Sicherheitskonzeptes die Same"=Origin"=Policy. Diese bewirkt, dass Sprachen, die auf Client"=Seite ausgeführt werden (wie Javascript), nicht die Möglichkeit haben, Requests, also Anfragen, an einen anderen Zielpunkt als ihrem Ursprung zu starten (vgl. \cite{Ruderman2008}). Diese Policy wird also lediglich bei Zugriff auf \acp{URL} mit der selben Domain und dem selben Port, wie die \ac{URL} von der die Seite geladen wurde, erfüllt. Das bedeutet, dass ein Skript auf \path{http://sop.example.com/directory1} Requests an \path{http://sop.example.com/directory2} starten kann, nicht jedoch an \path{http://example.com/directory2} (unterschiedliche Domain) oder an \path{http://sop.example.com:8080/directory2} (unterschiedlicher Port). Ausgenommen ist hierbei das in eine Seite eingebettete Laden von Ressourcen. Hierzu gehören externe Inhalte, die über iframes geladen werden aber auch externe Javascript"=Dateien (über $<$script$>$...$<$/script$>$ Tags) und Medienressourcen wie Bilder und Videos. Des Weiteren ist es auch möglich Formulare an andere Zielpunkte als den Ursprung abzuschicken. Diese Einschränkung hat also primär Auswirkungen auf das Absenden von XMLHttp"=Requests, also auf normale Ajax"=Requests.

Die Same"=Origin"=Policy ist sehr sinnvoll, um beispielsweise gefährlichem Javascript"=Code das Ausspähen privater Daten zu verhindern. Sie erschwert jedoch die Entwicklung moderner Ajax"=Anwendungen und insbesondere die Entwicklung von Mashup"=Applikationen wie \acp{PLE}, welche prinzipiell so aufgebaut sind, dass sie ihre Inhalte und Ressourcen aus unterschiedlichen Quellen beziehen. In einer \ac{PLE} wie in dieser Arbeit beschrieben, werden die Widgets von einem Widget"=Container wie Wookie (siehe Abschnitt \ref{section:widget_frameworks}) ausgeliefert und besitzen dadurch auch die Domain des Widget Containers als Origin. Arbeiten diese Widgets nun aber nicht nur lokal beim Client, sondern benötigen für ihre Funktionalität auch externe Server, so müssen sie in der Lage sein XMLHttp"=Requests an diese zu senden. Aus diesem Grund wurde der Mechanismus des \ac{CORS} (vgl. \cite{vanKesteren2012}) eingeführt. Dieser erlaubt es unter bestimmten Bedingungen und Einschränkungen die Same"=Origin"=Policy zum umgehen.

\subsubsection*{\acs{CORS}}
Wenn \acl{CORS} benutzt wird, kann der Server so konfiguriert werden, dass er Anfragen von anderen Origins erlaubt und angibt welche Header in diesen Anfragen vorkommen dürfen. Ein einfacher \ac{CORS}"=Request vom Client zum Server sieht wie folgt aus (Workflow analog zu \cite{Hossain2012}):
Der Client sendet eine Cross"=Origin"=Anfrage mit einem Origin Header an den Server:
\begin{lstlisting}
GET /cors HTTP/1.1
Origin: http://api.bob.com
Host: api.alice.com
\end{lstlisting}
Anschließend antwortet der Server mit:
\begin{lstlisting}
Access-Control-Allow-Origin: http://api.bob.com
Access-Control-Allow-Credentials: true
Access-Control-Expose-Headers: FooBar
\end{lstlisting}
Alle für den \ac{CORS}"=Request relevanten Header beginnen mit Access"=Control.\\ \texttt{Access\allowbreak -Control\allowbreak -Allow\allowbreak -Origin} bedeutet, dass der Server eine Cross"=Origin"=Anfrage von dem angegebenen Origin erlaubt, \texttt{Access\allowbreak -Control\allowbreak -Allow\allowbreak -Credentials: true} besagt, dass in diesem Request auch Cookies erlaubt sind. Möchte der Client Zugriff auf Nicht"=Standard"=Header aus der Antwort des Servers, müssen diese in \texttt{Access\allowbreak -Control\allowbreak -Expose\allowbreak -Headers} angegeben werden.

Sollte der Client einen Request mit einer anderen Methode als \texttt{GET} oder \texttt{POST} (siehe Abschnitt \ref{section:rest}) senden, reicht dieser einfache Workflow nicht aus. In diesem Fall muss vor der eigentlichen Anfrage ein so genannter "`Preflight"=Request"' ablaufen, welcher verifiziert, dass der Server diese Methode als \ac{CORS}"=Request erlaubt.

Diese Methodik erlaubt also, das Widgets von einem Widget"=Container ausgeliefert werden, dadurch einen anderen Origin bekommen und trotzdem Requests an ihr eigentliches Backend senden zu können. Damit ist sie ein essentieller Baustein für den Aufbau von Mashup"=Anwendungen im Internet.

\subsubsection*{Postmessage}
Für die in Anforderung \reqref{requirementWidgetInformSystem} geforderte Darstellung der wichtigsten Informationen über den Zustand eines Widgets ist es notwendig, dass die Hauptapplikation Daten von den Widgets erhalten oder auslesen kann. Mashup"=Anwendungen im Internet werden, wie in Kapitel \ref{section:widgets} beschrieben, hauptsächlich über iframes umgesetzt. Die Same"=Origin"=Policy aktueller Browser erlaubt aber nur, dass die Hauptanwendung lediglich Daten aus iframes mit der gleichen Origin ausliest. Dies erschwert den Informationsaustausch zwischen der Hauptapplikation und den Widgets beträchtlich. Aus diesem Grund wurde das "`Postmessage"' System entwickelt, welches diese Einschränkung auf eine sichere Art und Weise umgeht. Mit Hilfe von Postmessage ist es möglich, dass sich unterschiedliche Fenster (zum Beispiel Frames) Nachrichten schicken können. Andere Fenster können Event"=Listener implementieren, welche auf diese Nachrichten hören (vgl. \cite{MDN2012}). Im folgenden Listing wird deutlich, wie solch eine Nachricht versendet wird (aus \cite{MDN2012}):
\begin{lstlisting}
otherWindow.postMessage(message, targetOrigin);
\end{lstlisting}
\texttt{otherWindow} ist hierbei eine Referenz auf ein anderes Fenster oder einen anderen Frame, welches über einfache \ac{DOM}"=Methoden zu erreichen ist. \texttt{message} ist die Nachricht selber (kann auch ein Javascript"=Objekt sein) und \texttt{targetOrigin}, ist die Origin, die \texttt{otherWindow} haben muss, damit die Nachricht verschickt wird. Damit \texttt{otherWindow} die Nachricht empfängt, muss auf dieser Seite ein Event Listener implementiert werden:
\begin{lstlisting}
window.addEventListener("message", receiveMessage);
function receiveMessage(event) {
  if (event.origin !== "http://example.org:8080")
    return;
  // ...
\end{lstlisting}
Empfängt das Fenster also eine Nachricht ist es möglich auch den Origin des Senders zu prüfen, damit man nur Nachrichten aus vertrauenswürdigen Quellen bearbeitet.

Mit dieser Technik ist es möglich, dass sich Widgets bei der Hauptapplikation registrieren und ihr Informationen, z.B. über die Anzahl der vorhandenen Einträge, zukommen lassen. Diese Informationen kann die Hauptapplikation dann für den Anwender aufbereiten und darstellen, so dass sich Anforderung \reqref{requirementWidgetInformSystem} erfüllen lässt.

\subsection{\acs{REST}}\label{section:rest}
\ac{REST}, ist ein Architekturstil, welcher versucht 
\begin{quotation}
[...] für statische Inhalte und dynamisch berechnete Informationen, die ein globales, gigantisches Informationssystem bilden, ein einheitliches Konzept zu definieren (\cite{tilkovrest}, Seite 10).
\end{quotation}
Wird ein System nach dem \ac{REST}-Prinzipien entworfen, führt dies im Idealfall zu positiven Eigenschaften, wie der losen Kopplung von Systemen, dem Vorhandensein allgemeiner Interfaces und einer dadurch erhöhten Interoperabilität, Wiederverwendbarkeit und Erweiterbarkeit (vgl. \cite {tilkovrest}).

Theoretisch ist es möglich ein beliebiges \ac{REST}"=konformes Protokoll zu entwickeln und anzuwenden, normalerweise wird jedoch das im Internet dominierende \ac{HTTP}"=Protokoll für die Umsetzung benutzt. Der Grund hierfür ist, dass dieses Protokoll bei korrekter Anwendung die wichtigsten Anforderungen des \ac{REST}"=Prinzips erfüllt und schon im Hinblick auf diesen Architekturstil entworfen wurde (einer der Hauptbeteiligten am Entwurf der \ac{HTTP}"=Spezifikation, Roy Fielding, hat mit seiner Dissertation :"`Architectural Styles and the Design of Network"=based Software Architectures"' das \ac{REST}"=Prinzip entwickelt (vgl. \cite{Fielding2000})). 
In einem \ac{REST}"=Entwurf wird nach Entitäten gesucht, welche auf Ressourcen abgebildet werden. Dies ähnelt einem objektorientierten Entwurf, es gibt hierbei jedoch zwei wichtige Unterschiede. Es werden den Ressourcen keine neuen Methoden oder Verben hinzugefügt, sondern es werden nur die Standardmethoden genutzt. Um trotzdem keine Einschränkung in der Mächtigkeit dieses Prinzips in Kauf nehmen zu müssen, werden mehr Ressourcen hinzugefügt, als es eigentlich Entitäten gibt. In einem dem \ac{REST}"=Prinzipien folgenden System speichert der Server keinerlei Session Informationen über den Nutzer. Dies hat zur Folge, dass beispielsweise der Warenkorb in einem Shop ebenfalls als Ressource angelegt werden muss. Der Vorteil von dieser Vorgehensweise ist der, dass somit z.B. auch Lesezeichen auf einen Warenkorb angelegt werden können. Zusätzlich gibt es beispielsweise noch neben den Primärressourcen, welche einzelnen Entitäten abbilden, Listenressourcen welche für die Auflistung einer Menge von Ressourcen eines bestimmten Typen genutzt werden (vgl. \cite{tilkovrest}). Im Internet besteht bereits ein eindeutiges System für die öffentliche Vergabe von Ressourcen"=Identifizierern: die \ac{URI}, z.B. \path{http://example.com/accounts/5?paginate=1&page=3}). URIs stellen einen allgemein verfügbaren Namensraum zur Verfügung, in dem Ressourcen identifiziert werden können. \ac{REST} fordert, dass Ressourcen in unterschiedlichen Repräsentationen ausgeliefert werden können. So ist es beispielsweise möglich, dass es eine maschinenlesbare Repräsentation im \ac{XML}"=Format und eine für Menschen im Browser darstellbare Repräsentation im \ac{HTML} Format gibt. Der Client kann über einen Accept"=Header ein bestimmtes Format anfordern.
Möchte man zum Beispiel einen Kunden über eine Ressource abbilden könnte der Identifizierer \path{http://example.com/customers/4} sein. Eine Anfrage der Darstellung kann dann als einfacher \ac{HTTP}"=Request abgeschickt werden:
\begin{lstlisting}
GET /customers/4 HTTP/1.1
Host: example.com
Accept: application/xml 
\end{lstlisting}
Der Server gibt in mit einem Statuscode (hier 200 OK) zurück, ob die Anfrage erfolgreich war und fügt dann die gewünschte Repräsentation der Ressource an:.
\begin{lstlisting}
HTTP/1.1 200 OK
...
[XML spezifische Kopfzeilen]
<customer href="./customers/4">
  <name>Roman Sachse</name>
  <orders>
    <order>
      <delivered>true</delivered>
      <link>http://example.com/orders/54>
\end{lstlisting}
In der Antwort des Servers finden sich Links auf weitere mit dem Kunden verknüpfte Ressourcen. Dies ist eine Umsetzung des Hypermedia"=Prinzips, welches für ein vollkommenes \ac{REST}"=System fordert (vgl. \cite{tilkovrest}), dass sich der Client ab einem Startpunkt aus nur durch das Verfolgen von Links durch das gesamte System bewegen könnte. Eine Listenressource würde man über \path{http://example.com/customers} abfragen, die Antwort könnte dann in \ac{XML}"=Verweise auf die einzelnen Customer erhalten. Zusätzlich könnten bei gewünschter Paginierung der Ergebnisse (beispielsweise über Query"=Parameter wie \path{http://example.com/customers?page=2} noch Links auf die vorherige und nächste Seite zurückgegeben werden.

Wie man in der obigen Anfrage sieht, wird eine Anfrage mit dem Verb GET an den Server geschickt. Das \ac{HTTP}"=Protokoll stellt unterschiedliche Standardmethoden bereit, mit denen ein Client eine Ressource abfragen kann. Der Vorteil bei der Nutzung dieser Standardmethoden liegt darin begründet, dass mit ihnen ein Interface existiert, mit dem alle Ressourcen abgefragt werden können und die ein Server für alle Ressourcen versteht. Sollte dies einmal nicht der Fall sein, kann der Server mit einer Fehlermeldung im Sinne von "`Methode wird für diese Ressource nicht unterstützt"' antworten. \ac{HTTP} stellt acht Methoden bereit, wobei hier aus Gründen des Umfangs nur die in der Praxis wichtigsten und auch in der Arbeit verwendeten fünf kurz vorgestellt werden (vgl. \cite{tilkovrest}):
\begin{enumerate}
 \item mit \texttt{GET} fordert der Client eine Repräsentation einer Ressource an.
 \item \texttt{PUT} aktualisiert oder erstellt (wenn nicht vorhanden) eine Ressource. 
 \item \texttt{POST} erstellt eine Ressource (wird an Listenressource gesendet).
 \item \texttt{DELETE} löscht eine Ressource.
 \item \texttt{OPTIONS} liefert Metadaten für eine Ressource - z.B. Allow"=Header, welche angeben, welche Methoden/Header akzeptiert werden (siehe Abschnitt \ref{section:same_origin_policy}).
\end{enumerate}

Um zu den Customer"=Ressourcen aus dem obigen Beispiel also einen Kunden hinzuzufügen, müsste man einen POST"=Request an die Listenressource \path{http.example.com/customers} senden. Ein einzelner Kunde kann mit einem GET"=Request an die Primärressource \path{http.example.com/customers/4} ausgelesen werden. PUT oder DELETE würden die Ressource dementsprechend aktualisieren oder löschen

GET und POST sind die am häufigsten genutzten Methoden. Der Grund hierfür ist, dass diese beiden Methoden der normale Weg eines Browser sind mit einem Server zu kommunizieren. Jeder direkte Aufruf einer URI über einen Browser sendet eine GET"=Anfrage an einen Server. Die im Web häufig anzutreffenden Formulare werden meist über eine POST"=Anfrage abgeschickt. So werden diese beiden Methoden ausserhalb von Anwendungen, die dem \ac{REST}"=Prinzip folgen, auch für alle anderen Anfragen benutzt, so dass man sich bei diesen Anwendungen nicht sicher sein kann, ob ein GET"=Request an eine URI nicht ein Ändern oder Löschen der Ressource zur Folge hat (man betrachte beispielsweise die URI: \path{http://example.com/customers/5?action=delete}).

Wird eine Anwendung im Hinblick auf \ac{REST}"=Prinzipien umgesetzt, so vereinfacht dies die Erweiterung dieses Systems beträchtlich (Anforderung \reqref{requirementExtensibility}). Insbesondere für Mashup"=Anwendungen bei denen unterschiedliche Systeme miteinander kommunizieren müssen, ist beispielsweise die Nutzung eines gemeinsamen Methodeninterfaces sehr hilfreich, da nicht für jedes System ein weiteres Kommunikationsprotokoll erstellt werden muss. Das System dieser Arbeit hat noch die zusätzliche Anforderung den Anwender in die Lage zu versetzen auch offline zu arbeiten (Anforderung \reqref{requirementOfflineWork}). Wenn das System für das Speichern der Daten im Browser ebenfalls auf \ac{REST} aufsetzt, wird es mit relativ wenig Aufwand möglich sein weitere offline fähige Widgets für unterschiedliche Services zu erstellen. Die Art des Services ist dabei egal, solange er in der Lage ist einfache \ac{REST}"=Anfragen zu verstehen, zu verarbeiten und standardkonform auf sie zu antworten. Für den Entwickler wird es in diesem Fall vollkommen transparent sein, ob sich das System in einem online oder offline ist. Er kann unabhängig von dem Zustand die Daten ohne zu implementierende Fallunterscheidungen speichern und abfragen.

\section{Eine \acs{PLE} auf dem USB-Stick}\label{section:ple_auf_usb}
Anforderung \reqref{requirementUsbStick} fordert die Möglichkeit das System auf unterschiedlichen Rechnern zu benutzen und die Daten offline zwischen diesen Rechnern auszutauschen. Da alle systemrelevanten Daten im Browser hinterlegt werden (Application"=Cache (siehe Abschnitt \ref{section:appcache}) und Web"=Storage (siehe Abschnitt \ref{section:offline_storage}), ist es notwendig die gesamte Browserinstallation zwischen unterschiedlichen Rechnern hin und herbewegen zu können. Der einfachste Weg hierfür ist die Installation des Browsers auf einem portablen Speichermedium wie einem USB"=Stick. Hierfür gibt es beispielsweise die Open"=Source"=Anwendung \href{http://portableapps.com/}{PortableApps}\footnote{weitere Informationen auf: \url{http://portableapps.com/}}, die genau dies erlaubt. Leider existieren solche Möglichkeiten bisher nur auf Windows-Systemen, so dass diese Anforderung nicht auf anderen Betriebssystemen wie zum Beispiel Distributionen auf Linux Basis zu erfüllen ist. 