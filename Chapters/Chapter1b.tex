\chapter{Einleitung} 
\label{chapter:Kapitel1}
\lhead{Kapitel 1. \emph{Einleitung}} 
\section{Einleitung}
Lebenslanges Lernen ist ein Konzept welches in unserer Gesellschaft immer mehr an Bedeutung gewinnt. Klassische Ansätze computergestützten Lernens, realisiert durch monolithische Lern-Management-Systeme können hier nicht immer weiterhelfen oder stehen dem sogar kontraproduktiv gegenüber. Aus diesem Grund ist in den letzten Jahren das Konzept der Personal Learning Environments immer stärker in den Fokus für computergestütztes Lernen gerückt. Personal Learning Environment sind Lernumgebungen, in denen der Anwender selber bestimmen kann welche Werkzeuge er für seine persönliche Art des Lernens nutzen möchte. Diese Umgebungen sind meist browserbasiert, laufen also komplett im Web und müssen nicht auf dem Computer des Anwenders installiert werden. 

In Zeiten von HTML5 und Web 2.0 stehen dem Nutzer online immer mehr Werkzeuge und soziale Netzwerke zur Verfügung, welche er für seine Lernaktivitäten benutzen kann. Leider steht dem Anwender jedoch nicht immer und zu jeder Zeit ein Internetzugang zur Verfügung. Die Grunde hierfür reichen von Netzabbrüchen bei der Nutzung von Smartphones oder Tablets bis hin zu mangelnder Infrastruktur in strukturell eher schwächeren Ländern.

Das Ziel dieser Arbeit ist die Entwicklung eines Prototypen einer leichtgewichtigen Lernumgebung auf Basis aktueller HTML5-Technologien. Diese Umgebung soll als Personal Learning Environment Verwendung finden. Möglich wird dies, indem die Anwendung als Aggregator fungiert, welcher Services unterschiedlichster Quellen auf einer oder mehrerer Seiten zusammenfasst und die Interaktion mit ihnen ermöglicht. Diese kleinen Anwendungen, sogennante Widgets, stellen eine Teilmenge der Funktionalitäten des eigentlichen Serivces zur Verfügung. Sie können Inhalte verschiedenster Services wie Twitter, Facebook, Etherpad lite oder auch Todo Listen darstellen und bearbeitbar machen. Somit wird die Lernumgebung zu einem Einstieg oder Portal in die Arbeit mit unterschiedlichsten Services. 

Der zentrale technische Aspekt der Arbeit ist die Offlinefähigkeit des Systems. Das heißt es soll möglich sein, zumindest in den wichtigsten Bereichen, mit dem System weiterzuarbeiten, auch wenn keine Internetverbindung besteht. Hat der Anwender erneut die Möglichkeit sich mit dem Netz zu verbinden, so sollen seine Änderungen mit den zugrunde liegenden Services synchronisiert werden.

Die Arbeit liefert als Ergebnis ein System, welches durch Designüberlegungen, APIs und prototypische Implementierungen darauf aufbauenden Arbeiten ein Fundament stellt, um es zu einer vollwertigen PLE auszubauen.

\subsection{Aufbau der Arbeit}
Kapitel \ref{chapter:Kapitel2} gibt als Hintergrundinformation eine kurze Einführung in die Thematik des computergestützten Lernens mit besonderem Blick Lern-Management-System und insbesondere auf Personal Learning Environments. Im folgenden Kapitel \ref{chapter:Kapitel3} werden die Anforderungen an die zu entwickelnde Anwendung anhand eines Use-Cases ermittelt und herausgearbeitet. Anschließend beschäftigt sich Kapitel \ref{chapter:Kapitel4} mit dem aktuellen Stand der Technik und der Forschung. Im technologischen Bereich werden die Technologien und Konzepte vorgestellt, welche für die Umsetzung der Anforderungen nötig sind. Anschließend werden zwei theoretische Konzepte besprochen, welche für die Einordnung und die Entwicklung von Personal Learning Environments von großem Nutzen sind. Abschließend werden kurz bestehende Systeme beschrieben und evaluiert. Kapitel \ref{chapter:Kapitel5} beschreibt ohne ins Detail zu gehen, das User Interface, welche der in Kapitel \ref{chapter:Kapitel4} beschriebenen Technologien wie zum Einsatz kommen und welche Konzepte zur Klassifizierung von Personal Learning Environments implementiert wurden. Daran anschließend gibt Kapitel \ref{chapter:Kapitel6} einen genaueren Einblick in Designentscheidungen und Implementierungsdetails. Des weiteren wird kurz auf die bei der Umsetzung verwendeten Werkzeuge und Frameworks eingegangen. Kapitel \ref{chapter:Kapitel7} beginnt mit einer Zusammenfassung der erarbeiteten Lösung und schließt mit einem, Aufgrund der Prototypenhaftigkeit des entwickelten System, etwas längerem Ausblick auf mögliche Weiterentwicklungen und Verbesserungen.
