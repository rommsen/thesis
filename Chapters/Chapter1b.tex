\chapter{Einleitung} 
\label{chapter:Kapitel1}
\lhead{Kapitel 1. \emph{Einleitung}} 
\section{Motivation}
Durch die veränderten Anforderungen an Studenten (in dieser Arbeit wird im Folgenden die maskuline Form aus Gründen der besseren Lesbarkeit gewählt), Auszubildende und spätere Berufstätige gewinnt das lebenslange Lernen eine immer wichtigere Bedeutung in unserer Gesellschaft. Arbeitnehmer und Lernende müssen sich in ihrem Berufsleben ständig weiterbilden und selbständig auf neue Gegebenheiten einstellen. Dies erfordert eine individuelle Organisation der zur Verfügung stehenden Informationen. In Zeiten von HTML5 und Web 2.0 stehen dem Nutzer online immer mehr Werkzeuge und soziale Netzwerke zur Verfügung, welche er für seine persönlichen Lernaktivitäten nutzen kann. Zur Arbeit mit diesen Werkzeugen benötigt der Anwender Systeme, die ihm helfen diese zu organisieren und in einen sinnvollen Lernkontext einzufügen. Klassische Ansätze computergestützten Lernens, realisiert durch monolithische Lern-Management-Systeme, können dem Anwender meist nur in institutionellen Kontexten weiterhelfen, da sie ihren Fokus primär auf konservative und kurszentrierte Lernmethodiken setzen. Dies reicht jedoch nicht aus, um den veränderten Lernbedingungen gerecht zu werden. Aus diesem Grund ist in den letzten Jahren das Konzept der Personal Learning Environments (PLEs) immer stärker in das Blickfeld der Forschung im Bereich des computergestützten Lernens gerückt. Personal Learning Environments sind Lernumgebungen, in denen der Anwender selbst bestimmen kann, welche Werkzeuge er für seine persönliche Art des Lernens nutzen möchte. Diese Umgebungen sind meist browserbasiert, laufen also komplett im Web und müssen nicht auf dem Computer des Anwenders installiert werden. 
Leider steht dem Anwender jedoch nicht immer und zu jeder Zeit ein Internetzugang zur Verfügung. Die Gründe hierfür reichen von Netzabbrüchen bei der Nutzung von Smartphones oder Tablets bis hin zu mangelnder Infrastruktur in strukturell eher schwächeren Ländern. Daher ist es wichtig, dass ein Anwender eine Lernumgebung auch dann benutzen kann, wenn einmal keine Internetverbindung zur Verfügung steht. Die dargestellte Problematik verdeutlicht, dass Lösungen entwickelt werden müssen, welche in der Lage sind auf die individuellen Anforderungen der Lernenden zu reagieren und ihnen in den unterschiedlichsten Situationen (z.B. wenn keine Internetverbindung besteht) zur Verfügung zu stehen. Diese Arbeit soll Ansätze für die Umsetzung solcher Systeme liefern.

\section{Aufgabenstellung}
Das Ziel dieser Arbeit ist die Planung und Entwicklung eines lauffähigen Prototypen einer leichtgewichtigen Lernumgebung auf Basis aktueller Technologien. Diese Umgebung soll als Personal Learning Environment Verwendung finden. Die Anwendung soll als Aggregator dienen, welche Services unterschiedlichster Quellen auf einer oder mehrerer Seiten zusammenfasst und die Interaktion mit ihnen ermöglicht. Für die Umsetzung des Prototypen soll ein User-Interface entworfen und implementiert werden, welches in aktuellen Browsern ohne zusätzlichen Installationsaufwand lauffähig ist. Daneben besteht die zentrale technische Anforderung der Arbeit darin, eine Möglichkeit zu finden, ein offline fähiges System zu entwickeln. Dies bedeutet, dass es möglich sein soll, zumindest in den wichtigsten Bereichen mit dem System weiterzuarbeiten, auch wenn keine Internetverbindung besteht. Hat der Anwender erneut die Möglichkeit sich mit dem Netz zu verbinden, sollen seine Änderungen automatisch mit den zugrunde liegenden Services synchronisiert werden.

Die Arbeit soll als Ergebnis ein System liefern, welches durch Überlegungen, APIs und prototypische Implementierungen darauf aufbauenden Arbeiten ermöglicht, eine vollwertige Personal Learning Environment aus dem gelieferten Fundament zu erstellen.

\section{Aufbau der Arbeit}
Kapitel \ref{chapter:Kapitel2} - "`\nameref{chapter:Kapitel2}"' legt eine Grundlage für die Arbeit. Es gibt eine kurze Einführung in die Thematik des computergestützten Lernens mit besonderem Blick Lern-Management-Systeme. Anschließend wird über die Einführung der Lerntheorie des Konnektivismus eine Brücke zu Personal Learning Environments geschlagen. Das Kapitel schließt zum Ziele der Begriffsklärung mit einem kurzen Blick auf Mashup-Anwendungen, Widgets und Workspaces. Um die Anforderungen an das zu entwickelnde System herauszuarbeiten wird im folgenden Kapitel \ref{chapter:Kapitel3} - "`\nameref{chapter:Kapitel3}"' ein Szenario entworfen, in dem eine Personal Learning Environment sinnvoll zum Einsatz kommen kann. Auf Grundlage dieses Szenarios werden die funktionalen und nichtfunktionalen Anforderungen an die zu entwickelnde Anwendung mit Hilfe von Anwendungsfällen und Randbedingungen abgeleitet und ermittelt. Kapitel \ref{chapter:Kapitel4} - "`\nameref{chapter:Kapitel4}"' beschäftigt sich darauf aufbauend mit dem für die Erfüllung der Anforderungen relevanten aktuellen Stand der Technik und der Forschung. Es werden zwei theoretische Konzepte besprochen, welche für die Einordnung und die Entwicklung von Personal Learning Environments von großem Nutzen sind. Im technologischen Teil werden Technologien und Konzepte für die Umsetzung einer PLE als Mashup-Anwendung beschrieben und auf Grundlage der Anforderungen evaluiert. Abschließend werden kurz ähnliche Systeme vorgestellt und mit dem vorliegenden Ansatz verglichen, um das System im aktuellen Markt besser einordnen zu können. Kapitel \ref{chapter:Kapitel5} - "`\nameref{chapter:Kapitel5}"' gibt einen Überblick über die konzeptuelle Lösung des Problems. Es stellt die wichtigsten Komponenten und deren Zusammenspiel, sowie das entworfene und umgesetzte User-Interface der PLE vor. Daran anschließend gibt Kapitel \ref{chapter:Kapitel6} - "`\nameref{chapter:Kapitel6}"' einen genaueren Einblick in Designentscheidungen und Implementierungsdetails. Des Weiteren wird kurz auf die bei der Umsetzung verwendeten Werkzeuge und Frameworks eingegangen. Kapitel \ref{chapter:Kapitel7} - "`\nameref{chapter:Kapitel7}"' fasst die Ergebnisse der vorliegende Arbeit zusammen und schließt mit einem Ausblick auf mögliche Weiterentwicklungen und Verbesserungen.
