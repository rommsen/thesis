\chapter{Einleitung} 
\label{chapter:Kapitel1}
\lhead{Kapitel 1. \emph{Einleitung}} 
\section{Motivation}
Lebenslanges Lernen ist ein Konzept welches in unserer Gesellschaft immer mehr an Bedeutung gewinnt. Klassische Ansätze computergestützten Lernens, realisiert durch monolithische Lern-Management-Systeme können hier nicht immer weiterhelfen oder stehen dem sogar kontraproduktiv gegenüber. Aus diesem Grund ist in den letzten Jahren das Konzept der Personal Learning Environments immer stärker in den Fokus für computergestütztes Lernen gerückt. Personal Learning Environment sind Lernumgebungen, in denen der Anwender selber bestimmen kann welche Werkzeuge er für seine persönliche Art des Lernens nutzen möchte. Diese Umgebungen sind meist browserbasiert, laufen also komplett im Web und müssen nicht auf dem Computer des Anwenders installiert werden. 

In Zeiten von HTML5 und Web 2.0 stehen dem Nutzer online immer mehr Werkzeuge und soziale Netzwerke zur Verfügung, welche er für seine Lernaktivitäten benutzen kann. Leider steht dem Anwender jedoch nicht immer und zu jeder Zeit ein Internetzugang zur Verfügung. Die Grunde hierfür reichen von Netzabbrüchen bei der Nutzung von Smartphones oder Tablets bis hin zu mangelnder Infrastruktur in strukturell eher schwächeren Ländern.

\subsection{Aufgabenstellung}
Das Ziel dieser Arbeit ist die Entwicklung eines lauffähigen Prototypen einer leichtgewichtigen Lernumgebung auf Basis aktueller HTML5-Technologien. Diese Umgebung soll als Personal Learning Environment Verwendung finden. Die Anwendung soll in der Lage sein als Aggregator zu fungieren, welcher Services unterschiedlichster Quellen auf einer oder mehrerer Seiten zusammenfasst und die Interaktion mit ihnen ermöglicht. Für die Umsetzung des Prototypen soll ein Userinterface entworfen und implementiert werden, welches in aktuellen Browsern ohne zusätzlichen Installationsaufwand lauffähig ist. Daneben besteht die zentrale technische Anforderung der Arbeit darin eine Möglichkeit zu finden das System offlinefähig zu machen. Dies bedeutet, dass es möglich sein soll, zumindest in den wichtigsten Bereichen, mit dem System weiterzuarbeiten, auch wenn keine Internetverbindung besteht. Hat der Anwender erneut die Möglichkeit sich mit dem Netz zu verbinden, so sollen seine Änderungen mit den zugrunde liegenden Services synchronisiert werden.

Die Arbeit soll als Ergebnis ein System liefern, welches durch Überlegungen, APIs und prototypische Implementierungen darauf aufbauenden Arbeiten ermöglicht auf Basis des erstellten Fundamentes eine vollwertige Personal Learning Environment zu erstellen.

\subsection{Aufbau der Arbeit}
Kapitel \ref{chapter:Kapitel2} gibt als Hintergrundinformation eine kurze Einführung in die Thematik des computergestützten Lernens mit besonderem Blick Lern-Management-Systeme. Es schlägt anschließend über die Einführung der Lerntheorie des Konnektivismus eine Brücke zu Personal Learning Environments und schließt mit einem kurzen Blick auf Widges. Im folgenden Kapitel \ref{chapter:Kapitel3} werden die funktionalen und nichtfunktionalen Anforderungen an die zu entwickelnde Anwendung anhand unterschiedlicher Use-Cases und Randbedingungen ermittelt. Anschließend beschäftigt sich Kapitel \ref{chapter:Kapitel4} mit dem aktuellen Stand der Technik und der Forschung. Im technologischen Bereich werden die Technologien und Konzepte vorgestellt, welche für die Umsetzung der Anforderungen nötig sind. Anschließend werden zwei theoretische Konzepte besprochen, welche für die Einordnung und die Entwicklung von Personal Learning Environments von großem Nutzen sind. Abschließend werden kurz bestehende Systeme beschrieben und evaluiert. Kapitel \ref{chapter:Kapitel5} beschreibt ohne ins Detail zu gehen, das User Interface, welche der in Kapitel \ref{chapter:Kapitel4} beschriebenen Technologien wie zum Einsatz kommen und welche Konzepte zur Klassifizierung von Personal Learning Environments implementiert wurden. Daran anschließend gibt Kapitel \ref{chapter:Kapitel6} einen genaueren Einblick in Designentscheidungen und Implementierungsdetails. Des weiteren wird kurz auf die bei der Umsetzung verwendeten Werkzeuge und Frameworks eingegangen. Kapitel \ref{chapter:Kapitel7} beginnt mit einer Zusammenfassung der erarbeiteten Lösung und schließt mit einem, Aufgrund der Prototypenhaftigkeit des entwickelten System, etwas längerem Ausblick auf mögliche Weiterentwicklungen und Verbesserungen.
