\chapter{Einleitung} 
\label{Kapitel 1}
\lhead{Kapitel 1. \emph{Einleitung}} 
\section{Einleitung}
Das Ziel dieser Arbeit ist die Entwicklung eines Prototypen einer leichtgewichtigen Lernumgebung auf Basis der aktuellen HTML5-Technologien. Diese Umgebung soll als Personal Learning Environment (PLE) Verwendung finden. Möglich wird dies, indem die Umgebung als Dashboard fungiert, welches Programme und Services unterschiedlichster Quellen, sogenannte Widgets, auf einer oder mehrerer Seiten zusammenfasst und die Interaktion mit ihnen ermöglicht. Diese Widgets sollen Inhalte von Services wie Twitter, Facebook, Etherpad lite oder auch Todo Listen darstellen und bearbeitbar machen. Somit wird das Dashboard zu einem Einstieg oder Portal in die Arbeit mit unterschiedlichsten Services. 
Die Arbeit betrachtet primär die Screen- und die Temporal-Dimension nach Palmer, wird sich aber auch, zumindest teilweise, mit der Data-Dimension befassen. Im ersten Teil der Arbeit wird die Screen-Dimension betrachtet. Dies bedeutet, dass das User-Interface des Dashboards auf Grund vorheriger Überlegungen und Klassifizierungen designed und prototypisch umgesetzt wird. Der zweite Teil der Arbeit befasst sich mit der Temporal-Dimension. Hier geht es um die Planung und Implementierung unterschiedlichster Mechanismen, um das System auch ohne einen ständigen Internetzugang nutzbar zu machen.
Das Ergebnis der Arbeit soll ein System liefern, welches durch Designüberlegungen, Apis und prototypische Implementierungen darauf aufbauenden Arbeiten das Fundament liefert es zu einer vollwertigen PLE auszubauen. In dieser finalen Version sollen dann alle 6 Dimensionen beachtet und implementiert werden.