\chapter{Hintergrund/Begriffsklärung} 
\label{Kapitel 2}
\lhead{Kapitel 2. \emph{Hintergrund/Begriffsklärung}} 
\section{Personal Learning Environments}
\subsection{Computergestütztes Lernen}\label{section:e-learning}
Computergestütztes Lernen (E-Learning) wird in Universitäten und anderen Bildungseinrichtungen zum großen Teil über zentralisierte Lern-Management-Systeme (LMS) durchgeführt. Diese ermöglichen es den Lehrenden Kurse zu Erstellen und Informationen, Zeitpläne, Dateien und andere Ressourcen zu diesen Kursen hinzuzufügen. Studenten können sich für anschließend für diese Kurse einschreiben. Dieses Schema ist momentan der Status Quo in den meisten höheren Bildungseinrichtungen \cite{Mott2010}. 

\subsection{Probleme mit Lern-Management-Systemen}
Die zentralisierte Art von LMS hat jedoch im Laufe der Zeit einige Schwächen offenbart. Nach Mott werden LMS zum Großteil für administrative Aufgaben eingesetzt (z.B. für Zeitplanung oder für das Hochladen von PDF-Dokumenten oder Vorlesungsfolien) und nicht für aktive oder interaktive Lernprozesse \cite{Mott2010}. Des weiteren sind LMS sehr kurs- und lehrerzentriert. Der Lehrnende wird primär als Konsument aufgefasst. Er kann sich beispielsweise bei Kursen anmelden, Hausaufgaben einreichen und sich über Resultate informieren. Aktive Beteiligung ist ihm möglich, jedoch nur in dem von dem Lehrenden vorgebebenen Rahmen.
Schließlich gibt es noch das Problem, dass der Zugriff auf die in den Systemen verteilten Daten auf mehrere Art und Weisen für den Nutzer beschränkt ist. Zum einen stehen die Kurse meist nur stehen den Studenten während einer definierten Zeit (z.B. einem Semester) offen. Nach Ablauf des Zeitraums wird der Zugriff eingeschränkt oder komplett aufgehoben. Zum anderen ist es so, dass dem Nutzer nach Verlassen der Bildungseinrichtung der Zutritt zu dem LMS oft komplett verweigert wird und er so keine Möglichkeit mehr hat auf die Daten und Werkzeuge innerhalb des LMS zuzugreifen \cite{Schaffert2008a}.

\subsubsection{Lebenslanges Lernen}
Der zeitlich eingeschränkte Zugriff auf Informationen steht dem Konzept des lebenslangen Lernens gegenüber. Lebenslanges Lernen ist eine Sichtweise, die es dem Menschen ermöglichen soll sich während seines gesamten Lebens selbständig neue Fähigkeiten anzueignen, also zu Lernen. Nach dem Bundesministerium für Bildung und Forschung hört "`Lernen [...] nach Schule, Ausbildung oder Studium nicht auf, denn Lernen ist das wesentliche Werkzeug zum Erlangen von Bildung und damit für die Gestaltung individueller Lebens- und Arbeitschancen. Lebenslanges Lernen heißt das Schlüsselwort, wenn man auf dem Arbeitsmarkt mithalten, einen Berufs- oder Schulabschluss nachholen oder sich einfach nur weiterbilden will"' \cite{bmbf_lebenslanges_lernen}. Aus diesem Grund benötigen auch Arbeiter kontinuierliches Lernen während ihrer gesamten Arbeitslebens, um ihrer berufsbezogenen Fähigkeiten zu aktualisieren und sich neue benötigte Fähigkeiten anzueignen \cite{Attwell2007}. Lern-Management-Systeme, die den Nutzer nach einiger Zeit ausschließen oder ihm Informationen verwehren passen nicht in dieses Konzept.

\subsubsection{Technologische Entwicklung}\label{section:technologische_entwicklung}
Die Technologien, die den am Lehrbetrieb Beteiligten (Lehrenden und Lernenden) für ihre Arbeitsabläufe zur Verfügung stehen in den letzten Jahren extrem geändert und weiterentwickelt. Das Aufkommen des Web 2.0 hat eine Vielzahl von Services und sozialen Netzwerken hervorgebracht, welche es den Nutzern erlauben aktiv in die Gestaltung von Inhalten und Ressourcen einzubringen. Hierzu gehören Wikis, Blogs, soziale Netzwerke wie Facebook, Webservices unterschiedlichster Arten (für Notizen, Todo-Listen, Kalender) etc..
Diese Entwicklung müsse sich nach Attwell auch im Bereich des E-Learnings widerspiegeln. Computergestütztes Lernen dürfe in Zeiten von Web 2.0 nicht mehr einfach nur die Formen des Lernens in Klassenräumen und Universitäten auf Software übertragen \cite{Attwell2007}. Es müsse dem Lehrnenden die Möglichkeit geben sich aktiv in den Lernprozess einzubringen und die von ihm genutzten Werkzeuge selber zu bestimmen. Der Lerner soll sich von einem reinen "`Consumer"' zu einem "`Prosumer"' entwickeln, also zu jemandem der gleichzeitig Konsument ("`Consumer"') und Produzent ("`Producer"') von Lernmaterialen ist \cite{Schaffert2008a}.  

\subsection{Personal Learning Environments - Einführung}
Personal Learnings Environments (PLEs) sind ein Konzept, um diese Forderungen umzusetzen und die oben beschriebenen Probleme von LMS zu vermeiden. \cite{Attwell2007}. PLEs nehmen Abstand von der kurs- oder institutszentrierten Sichtweise des computergestützten Lernens und propagieren eine nutzerzentrierte Sichtweise. Eine PLE soll dem Nutzer die Möglichkeit geben, dass System nach seinen Vorstellungen und Anforderungen zu personalisieren sich persönlich seine Werkzeuge, die er zum Lernen benötigt zusammenstellen und in einer Applikation zusammenfassen \cite{VanHarmelen}. Dies erlaubt es ihm diese auch bei sich veränderten Lebensumständen oder Lerngewohnheiten (wechseln der Bildungseinrichtung, des Arbeitsplatzes) anzupassen und weiter zu benutzen. Der Nutzer kann unabhängig vom Lehrinstitut Daten und Inhalte erstellen, welcher auch nach dem Verlassen des Instituts weiterhin zur Verfügung stehen. Er kann Lern- und Interessengruppen auch über Institutsgrenzen hinweg bilden \cite{Schaffert2008a} und er kann Werkzeuge wie Kalender, Todo-Listen oder Notizen nutzen und diese nach seinen eigenen Vorstellungen klassifizieren und sortieren.