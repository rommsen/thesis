\chapter{Hintergrund / Begriffsklärung} 
\label{chapter:Kapitel2}
\lhead{Kapitel 2 \emph{Hintergrund / Begriffsklärung}} 

Dieses Kapitel gibt eine Einleitung über den Hintergrund der Arbeit und führt für diese Arbeit zentrale Begriffe ein. Als Einstieg wird der Begriff des lebenslangen Lernens vorgestellt. Von dieser Basis aus wird eine Brücke von klassischen Lern-Management-Systemen und ihren Problemen mit dem lebenslangen Lernen über die Lerntheorie des Konnektivismus hin zur Notwendigkeit von Personal Learning Environments geschlagen. Anschließend wird der für Personal Learning Environments zentrale Begriff der Mashup-Anwendungen thematisiert. Das Kapitel schließt mit einer Vorstellung des Konzeptes von Widgets in Mashup-Anwendungen und einer Einführung des für diese Arbeit wichtigen Begriffes Workspace. 

\section{Lebenslanges Lernen}
Lebenslanges Lernen ist ein Konzept, welches es dem Menschen ermöglichen soll, sich während seines gesamten Lebens selbständig neue Fähigkeiten anzueignen. Nach dem Bundesministerium für Bildung und Forschung hört Lernen \begin{quotation}
 [...] nach Schule, Ausbildung oder Studium nicht auf, denn Lernen ist das wesentliche Werkzeug zum Erlangen von Bildung und damit für die Gestaltung individueller Lebens- und Arbeitschancen. Lebenslanges Lernen heißt das Schlüsselwort, wenn man auf dem Arbeitsmarkt mithalten, einen Berufs- oder Schulabschluss nachholen oder sich einfach nur weiterbilden will (\cite{BMBF2008}).
\end{quotation}
Aus diesem Grund benötigen nicht nur Studenten, sondern auch andere Menschen während ihrer gesamten Arbeitslebens die Möglichkeit zum kontinuierlichen Lernen, um ihre berufsbezogenen Fähigkeiten zu aktualisieren und sich neue benötigte Fähigkeiten anzueignen (vgl. \cite{Attwell2007}). Systeme zum computergestützten Lernen müssen sich folglich auf die Anforderungen einstellen.

\section{Lern-Management-Systeme}\label{section:lms}
Computergestütztes Lernen (E-Learning) wird in Universitäten und anderen Bildungseinrichtungen zum großen Teil über zentralisierte Lern"=Management"=Systeme (LMS) durchgeführt. Diese ermöglichen es den Lehrenden Kurse zu erstellen und Informationen, Zeitpläne, Dateien und andere Ressourcen zu diesen Kursen hinzuzufügen und zu veröffentlichen. Studenten können sich anschließend für diese Kurse einschreiben und erhalten so Zugriff auf diese Daten. Dieses Schema ist momentan der Status Quo in den meisten höheren Bildungseinrichtungen (vgl. \cite{Mott2010}). 

Die zentralisierte Art von LMS hat jedoch im Laufe der Zeit einige Schwächen offenbart. Nach \cite{Mott2010} werden LMS zum Großteil für administrative Aufgaben eingesetzt (z.B. für Zeitplanung oder für das Hochladen von PDF-Dokumenten und Vorlesungsfolien) und nicht für aktive oder interaktive Lernprozesse. Des Weiteren sind LMS sehr kurs- und lehrerzentriert. Der Lernende wird primär als Konsument aufgefasst. Er kann sich beispielsweise bei Kursen anmelden, Hausaufgaben einreichen und sich über seine Resultate informieren. Aktive Beteiligung ist ihm möglich, jedoch nur in dem von dem Lehrenden vorgegebenen Rahmen.
Schließlich gibt es noch das Problem, dass der Zugriff auf die in den Systemen verteilten Daten auf mehrere Art und Weisen für den Nutzer beschränkt ist. Zum einen stehen die Kurse meist nur den Studenten während einer definierten Zeit (z.B. einem Semester) offen. Nach Ablauf des Zeitraums wird der Zugriff eingeschränkt oder komplett aufgehoben. Zum anderen wird nach \cite{Schaffert2008a} dem Nutzer nach Verlassen der Bildungseinrichtung der Zutritt zu dem LMS häufig komplett verweigert, so dass er keine Möglichkeit mehr hat auf die Daten und Werkzeuge innerhalb des LMS zuzugreifen. Dieser zeitlich eingeschränkte Zugriff auf Informationen steht dem Konzept des lebenslangen Lernens hinderlich gegenüber.

Die Technologien, die den am Lehrbetrieb Beteiligten (Lehrenden und Lernenden) für ihre Arbeitsabläufe zur Verfügung stehen, haben sich in den letzten Jahren sehr stark verändert und weiterentwickelt. Die Entwicklung des Web 2.0 hat eine Vielzahl von Services und sozialen Netzwerken hervorgebracht, welche es den Nutzern erlauben Inhalte und Ressourcen aktiv zu gestalten, anstatt sie nur zu konsumieren. Hierzu gehören Wikis, Blogs, soziale Netzwerke wie Facebook und Webservices unterschiedlichster Arten (für Notizen, Todo-Listen, Kalender etc.).
Diese Entwicklung muss sich nach \cite{Attwell2007} auch im Bereich des E-Learnings widerspiegeln. Computergestütztes Lernen darf in Zeiten von Web 2.0 nicht mehr einfach nur die Formen des Lernens in Klassenräumen und Universitäten auf Software übertragen. Es muss dem Lernenden die Möglichkeit gegeben werden, sich aktiv in den Lernprozess einzubringen und die von ihm genutzten Werkzeuge selbst zu bestimmen. Der Lerner soll sich von einem reinen "`Consumer"' zu einem "`Prosumer"' entwickeln, also zu jemandem der gleichzeitig Konsument ("`Consumer"') und Produzent ("`Producer"') von Lernmaterialien ist (vgl. \cite{Schaffert2008a}).

\section{Konnektivismus}\label{section:konnektivismus}
Befindet sich ein Lernender auf dem Wege zu einem "`Prosumer"', so ändert sich damit nach \cite{Schaffert2008a} sein Lernverhalten. Um diesem neuen Verhalten und den neuen Möglichkeiten und Anforderungen des Lernens in einer immer stärker digitalisierten Welt Rechnung zu tragen, hat Siemens eine neue Lerntheorie mit dem Namen "`Konnektivismus"' begründet. Unter dem Ansatz des Konnektivismus (engl. connectivism) versteht man eine Lerntheorie, die auf die veränderten Anforderungen des digitalen Zeitalters reagiert. Im Zentrum des Konnektivismus steht die Loslösung von der Sichtweise des isolierten Individuums, die Bildung von Wissensnetzwerken und der damit verbundenen Vernetzung von Wissen (vgl. \cite{Siemens2004}). Folglich kann der Konnektivismus in Verbindung mit den Konzepten und Methoden des Web 2.0 und des e-Learning 2.0 gebracht werden.
Nach dem Begründer der Theorie sind klassische Lerntheorien (Behaviorismus, Kognitivismus und Konstruktivismus) nicht in der Lage die technologischen Entwicklungen und die damit verbundenen Folgen der steigenden Wissensflut und der zunehmend sinkenden Halbwertzeit des Wissens einzubeziehen und abzubilden. Des Weiteren werden die gesellschaftlichen Änderungen des Lernens (wie zum Beispiel der Bedeutungsgewinn des informellen und lebenslangen Lernens) und die damit einhergehenden Veränderungen für die Lehre nicht berücksichtigt (vgl. \cite{Siemens2004}).
Die Hauptaussage des Konnektivismus ist, die Begrenztheit der klassischen Lerntheorien zu beheben und zu beachten, dass das Lernen sich nicht nur innerhalb eines Individuums, sondern auch in Organisationen, Gemeinschaften (Communities) und anderen vernetzten Strukturen vollziehen kann. Daraus folgt, dass der Konnektivismus die Verflechtung von Informationen in Netzwerken in den Blick nimmt. Diese Wissensquellen (z.B. Daten, Bilder, Texte etc.) werden als Knoten bezeichnet, die sich zu einem Netzwerk verbinden. Der individuell Lernende hat nun die Aufgabe sich sein eigenes Wissen aus den verschiedenen Quellen zu generieren. Hierbei kann das Individuum zum einen die Hilfe Anderer, zum anderen auch unterschiedliche externe Quellen nutzen (wie z.B. wichtige Web Sites, Blogs, soziale Netzwerke oder auch akademische Artikel oder Konferenzmaterialien) beanspruchen (vgl. \cite{Siemens2004}). Folglich stellt sich jedes Individuum ein eigenes persönliches Netzwerk aus relevanten Daten und Informationen zusammen. 

\section{Personal Learning Environments}\label{section:ple_intro}
Wie aus den obigen Erläuterungen hervorgeht, müssen sich Systeme zum computergestützten Lernen den neuen Anforderungen stellen und daran anpassen. Aus diesem Grund wurde das Konzept der Personal Learning Environment (PLE) entwickelt. PLEs sollen dem Nutzer die Möglichkeit geben, seine Lernumgebung nach seinen Vorstellungen und Anforderungen zu personalisieren, sich also selbständig die Werkzeuge, die er zum Lernen benötigt zusammenstellen und in einer Applikation zusammenfassen (vgl. \cite{VanHarmelen}). PLEs nehmen Abstand von der kurs- oder institutszentrierten Sichtweise von Lern"=Management"=Systemen und stellen somit eine Umsetzung der im Konnektivismus geforderten nutzerzentrierten Sichtweise auf das Lernen in unserer heutigen Informationsgesellschaft dar. PLEs erlauben es dem Anwender seine Lernumgebung auch bei sich veränderten Lebensumständen oder Lerngewohnheiten (z.B. bei einem Wechsel der Bildungseinrichtung, des Arbeitsplatzes) anzupassen und weiter zu benutzen. Dem Anwender wird es in einer PLE freigestellt, welche Ressourcen er für seinen Lernaktivitäten benutzt. Er kann kann unabhängig vom Lehrinstitut Daten und Inhalte erstellen, welche auch nach dem Verlassen des Instituts weiterhin zur Verfügung stehen. Er kann Lern- und Interessengruppen auch über Institutsgrenzen hinweg bilden (vgl. \cite{Schaffert2008a}) und er kann Werkzeuge wie Kalender, Todo"=Listen oder Notizen als sogenannte Widgets einbinden (siehe Abschnitt \ref{section:widgets}) und diese nach seinen eigenen Vorstellungen klassifizieren und sortieren. Personal Learning Environments beschreiben also ein Konzept, welches die in Abschnitt \ref{section:lms} propagierten Forderungen nach einem Wandel des Lernenden von einem Consumer zu einem Prosumer umsetzt und so die oben beschriebenen Probleme von Lern"=Management"=Systemen vermeiden soll (vgl. \cite{Attwell2007}).

\section{Mashup-Anwendungen}\label{section:mashup_anwendungen}
Personal Learning Environments sind grundsätzlich als Mashup"=Anwendungen konzipiert, welche aus diesem Grunde im folgenden Abschnitt vorgestellt werden. Mashup"=Anwendungen sind Applikationen, bei denen unterschiedliche Anwendungen zu einer Anwendung zusammengefasst und von dieser Anwendung aus bedient werden können (siehe Abschnitt \ref{section:ple_intro}). Nach \cite{Soylu2011} gibt es zwei unterschiedliche Arten von Mashup"=Anwendungen: "`box type mashups"' und "`dashboard type mashups"'. "`Box type mashups"' sind Anwendungen, bei denen unterschiedliche Services grafisch und konzeptuell so in eine Applikation integriert werden, dass der Anwender in seinem Umgang mit der Applikation nur diese wahrnimmt und nicht bemerkt, dass er eigentlich mit unterschiedlichen Anwendungen arbeitet. "`Dashboard type mashups"' hingegen fassen die unterschiedlichen Anwendungen nur in einem einheitlichen Rahmen zusammen. Dies geschieht meist über so genannte Widgets. Diese kleinen Anwendungen, stellen eine Teilmenge der Funktionalitäten des eigentlichen Services zur Verfügung. Sie können Inhalte verschiedenster Services wie Twitter, Facebook oder auch Todo Listen darstellen und bearbeitbar machen. Somit wird die Lernumgebung zu einem Einstieg oder Portal in die Arbeit mit unterschiedlichsten Services.  Die Inhalte und das User-Interface dieser Widgets stammen von den eigentlichen Services und können sich somit sehr stark voneinander unterscheiden. 

\subsection{Widgets}\label{section:widgets}
Widgets sind kleine eigenständige ausführbare Applikationen, die auf HTML-Seiten oder dem Desktop eingebettet und ausgeführt werden können (vgl. \cite{Taraghi2010}). Es ist möglich Widgets mit unterschiedlichen Technologien zu erstellen (beispielsweise als Java-Applets oder als Flash-Anwendung), aktuell werden sie jedoch meist ebenfalls als HTML/Javascript-Anwendung implementiert. Nach einer Recommendation des World Wide Web Consortium (W3C) sind Widgets
\begin{quotation}[...] full-fledged client-side applications that are authored using Web standards such as HTML and packaged for distribution. They are typically downloaded and installed on a client machine or device where they run as stand-alone applications, but they can also be embedded into Web pages and run in a Web browser. Examples range from simple clocks, stock tickers, news casters, games and weather forecasters, to complex applications that pull data from multiple sources to be "`mashed-up"' and presented to a user in some interesting and useful way (\cite{W3C-11-2012}).\end{quotation}
In eine Webanwendung geladen werden Widgets im Normalfall über iframes, also über ein HTML-Element welches erlaubt autonome Seiten in das aktuelle Dokument zu laden (vgl. \cite{W3C1999}). Grundsätzlich kann jede eigenständige Web-Anwendung als Widget fungieren, es gibt jedoch Bestrebungen die Formate in denen Web-Widgets implementiert werden können zu standardisieren. Der Vorteil hiervon ist, dass mit einer Standardisierung eine höhere Portabilität der Widgets zwischen unterschiedlichen Systemen einhergeht und wiederkehrende Probleme wie Authentifizierung oder Widget-Auslieferung allgemein gelöst werden können. Aktuell gibt es zwei unterschiedliche Spezifikationen zur Standardisierung von Widgets und verwandter Technologien: "`OpenSocial Gadgets"' und "`W3C Widgets"' (siehe Abschnitt \ref{section:widget_frameworks}).

\subsection{Workspaces}
Bei der Arbeit mit einer PLE sollte es dem Anwender möglich sein in unterschiedlichen Kontexten zu arbeiten und trotzdem das selbe System zu benutzen. Hierfür benötigt er unterschiedliche Arbeitsbereiche, sogenannte Workspaces, auf denen er seine verwendeten Widgets anordnen und gruppieren kann. Es wäre beispielsweise möglich, dass sich der Anwender einen Workspace pro Unikurs oder für ein bestimmtes Themengebiet anlegt, um so eine Ordnung in die genutzten Widgets zu bringen. 