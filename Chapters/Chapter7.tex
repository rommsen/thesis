\chapter{Zusammenfassung / Ausblick}\label{chapter:Kapitel7}

\section{Zusammenfassung}
Diese Arbeit hat Personal Learning Environments als wichtiges Hilfsmittel für das Konzept des lebenslangen Lernens beschrieben. Das Ziel dieser Arbeit war die prototypische Implementierung einer leichtgewichtigen Personal Learning Environment mit Offline"=Funktionalitäten auf Basis aktueller Technologien. Kapitel \ref{chapter:Kapitel3} hat ein Szenario vorgestellt aus dem die funktionalen und nichtfunktionalen Anforderungen an solch ein System abgeleitet wurden. Das darauf und auf den in Kapitel \ref{chapter:Kapitel4} vorgestellten Technologien und Konzepten aufbauend entwickelte System "`Plesynd"' ist ein Informationsaggregator, welcher die Inhalte unterschiedlichster Services in übersichtlicher Art und Weise in Form von Widgets darstellen kann. Als besonderes Merkmal bietet Plesynd dem Anwender nach einmaligem Laden der Applikationsdaten die Möglichkeit auch dann weiterzuarbeiten, wenn keine Konnektivität mit dem Internet besteht. Wenn die Verbindung wieder hergestellt wurde, werden die Daten mit den zugrunde liegenden Services automatisch synchronisiert. Zur Umsetzung dieser Funktionalitäten wurden ausschließlich freie Webtechnologien verwendet. Bei der Implementierung wurde außerdem darauf geachtet, dass es klare Ansatzpunkte gibt, an denen Plesynd erweitert werden kann. Durch die Nutzung des W3C"=Widget"=Standards und dem Zurverfügungstellen eines für den Entwickler transparenten Mechanismus zur Online"=/Offline"=Speicherung und Synchronisation von Daten, ist es möglich einfach neue Widgets für andere Services Plesynd"=konform zu implementieren. Plesynd kann ohne Installationsaufwand in aktuellen Browsern genutzt werden. Nur wenn das System von einem mobilen Speichermedium wie einem USB"=Stick aus verwendet werden soll, ist die Installation eines weiteren Programms notwendig. Dann kann Plesynd samt der hinterlegten Daten allerdings auch offline zwischen unterschiedlichen Rechnern ausgetauscht werden.  Wie die Tabellen \ref{table:nichtfunktionale_anforderungen_validierung} und \ref{table:funktionale_anforderungen_validierung} in Abschnitt \ref{section:validierung_der_anforderungen} zeigen, wurden mit der Umsetzung von Plesynd bis auf die funktionale Anforderung \reqref{requirementKeinZugriffNachLogout} und die nichtfunktionale Anforderung \reqref{requirementUsbStick} alle Anforderungen vollständig erfüllt. Für die vollständige Erfüllung von \reqref{requirementKeinZugriffNachLogout} ist es notwendig, dass der Local"=Storage aller Widgets nach einem Ausloggen geleert wird. Für die vollständige Erfüllung von \reqref{requirementUsbStick} ist es notwendig, dass Werkzeuge entwickelt werden, die es erlauben, mobile Versionen von Browsern auf USB-Sticks auch auf Basis anderer Betriebssysteme als Windows zu installieren. Diese Aufgabe lag jedoch weit außerhalb des Rahmens der vorliegenden Arbeit.

Zusammenfassend haben Kapitel \ref{chapter:Kapitel1} und Kapitel \ref{chapter:Kapitel2} die Grundlage für die vorliegende Arbeit geliefert und eine Motivation gegeben, welche den Paradigmenwechsel von zentralisierten Lern"=Management"=Systemen hin zu Personal Learning Environments notwendig macht. Die Lerntheorie des Konnektivismus verlangt nach Systemen, die den Lernenden in den Mittelpunkt eines sich konstant verändernden Wissensnetzwerkes stellt. In dieser Arbeit wurde mit dem Entwurf und der Umsetzung von Plesynd die Grundlage für ein solches System geschaffen.

\section{Ausblick}
Da die Umsetzung von Plesynd eine prototypische Implementierung darstellt, gibt es an den unterschiedlichsten Stellen Ansatzpunkte für Weiterentwicklungen, welche in auf dieser Arbeit aufbauenden Arbeiten behandelt werden können. Im Folgenden werden einige davon kurz vorgestellt:

\begin{itemize}
 \item Entwicklung neuer Widgets, die ebenfalls die offline Funktionalitäten nutzen. Möglich wären beispielsweise ein Twitter oder Facebook Client, ein Chat"=Widget oder ein RSS"=Reader. Da die W3C"=Widgets nichts anderes als gepackte HTML"=Applikationen sind, sind den Möglichkeiten an dieser Stelle kaum Grenzen gesetzt.
 \item Zum aktuellen Zeitpunkt wird kaum davon gebraucht gemacht, dass die W3C"=Widgets und Wookie es erlauben pro Widgets Einstellungen zu hinterlegen. Mögliche Anwendungsfälle hierfür wären zum Beispiel die Menge der anzuzeigenden Daten in einer Todo"=Liste oder einem RSS"=Reader.
 \item Implementation von Technologien, die es dem Server erlauben von sich aus Informationen an Plesynd oder die Widgets zu senden (Stichwort: Websockets), um den Nutzer noch besser über neue Daten etc. zu informieren.
 \item Einsatz von Media"=Queries zur besseren Nutzbarkeit auf mobilen Endgeräten.
 \item Implementation von Mechanismen, die es den Widgets erlauben auch untereinander in Kontakt zu treten und Daten auszutauschen.
 \item Für eine vollwertige PLE sollten auch die übrigen Dimensionen nach Palmér implementiert werden. Es wäre beispielsweise möglich verstärkt Augenmerk auf die "`Social"=Dimension"', also auf den Einbau von Social"=Network Fähigkeiten zu legen. Es wäre auch sinnvoll Lernsequenzen innerhalb der PLE zu implementieren. Hierzu können unterschiedlichste Konzepte aus dem Bereich des E"=Learnings wie zum Beispiel die Learning Design Specification des IMS Global Learning Consortium (vgl. \cite{IMS2012}) oder Learning Object Metadata (LOM) (vgl. \cite{LOM2002}) umgesetzt werden und so die "`Activity"=Dimension"' formieren.
 \item Erweiterung von Wilsons "`Discourse Monitor"': Es wäre möglich für die einzelnen Widgets nicht nur eine Zusammenfassung der Anzahl der zur Verfügung stehenden Datensätze anzuzeigen, sondern auch so etwas wie ein Erfolgsmonitoring auf dem Dashboard anzubieten. Damit ist eine Darstellung der Art: "`Zu erledigen: 5 Aufgaben, heute schon erledigt: 3 Aufgaben, überfällig: 0 Aufgaben"' gemeint. Dies geht wahrscheinlich nur bei Widgets die klar in die “Manage Time and Effort”"=Klassifizierung von Wilson fallen und erfordert zusätzlich noch eine semantische Analyse der eingehenden Daten.
\end{itemize}

