\chapter{Use-Cases für eine PLE im Style einer technischen Dokumentation} \label{AppendixE}

\section*{Registrieren}
\textbf{use case} \emph{Registrieren}\\
\textbf{actors} Student\\
\textbf{precondition} Das System ist online, der Student ist nicht eingeloggt und es existiert mindestens ein Workspace.\\
\textbf{main flow} Der Student bekommt auf der Startseite die Möglichkeit sich für den Zugang zum System zu registrieren. Er füllt ein Formular mit seinen Daten (Vorname, Nachname, E-Mail-Adresse, Nutzername, Passwort) aus und schickt das Formular ab.\\
\textbf{postcondition} Der Student ist mit einer Nutzername-/Passwortkombination und seiner E-Mail-Adresse im System hinterlegt und kann sich somit in das System einloggen.\\
\textbf{exceptional flow} Nutzername oder Mail-Adresse bereits vorhanden. Nutzername und E-Mail-Adresse dürfen nur einmal im System vorkommen.\\
\textbf{postcondition} Der Student bekommt eine Fehlermeldung und muss seine eingetragenen Daten ändern.
 
\section*{Login}
\textbf{use case} \emph{Login}\\
\textbf{actors} Student\\
\textbf{precondition} Das System ist online, der Student ist nicht eingeloggt, ist aber in dem System registriert.\\
\textbf{main flow} Der Student bekommt auf der Startseite die Möglichkeit sich im System anzumelden. Er füllt ein Formular mit seiner Nutzername-/Passwortkombination aus und schickt das Formular ab.\\
\textbf{postcondition} Der Student ist im System eingeloggt und das System präsentiert ihm eine Übersicht über seine aktuellen Work"-spaces und die wichtigsten aggregierten Informationen dieser.

\section*{Logout}
\textbf{use case} \emph{Logout}\\
\textbf{actors} Student\\
\textbf{precondition} Das System ist online, der Student ist eingeloggt.\\
\textbf{main flow} Der Student wählt die Aktion "`Logout"' und wird anschließend auf die Login-Seite weitergeleitet.\\
\textbf{postcondition} Die aktuelle Session des Anwenders ist beendet und es ist ohne einen weiteren Login nicht möglich, auf die Daten des Anwenders zuzugreifen 

\section*{Workspace hinzufügen}
\textbf{use case} \emph{Workspace hinzufügen}\\
\textbf{actors} Student\\
\textbf{precondition} Das System ist online, der Student ist eingeloggt.\\
\textbf{main flow} Der Student wählt die Aktion "`Workspace hinzufügen"'. Es öffnet sich hierdurch ein neuer Bereich, welcher einen neuen Workspace repräsentiert. Der Student hat nun die Möglichkeit den Workspace nach seinen Wünschen anzupassen (extension point: Workspace bearbeiten).\\
\textbf{postcondition} Ein neuer Workspace wurde dem System des Studenten hinzugefügt.
 
\section*{Workspace bearbeiten}
\textbf{use case} \emph{Workspace bearbeiten}\\
\textbf{actors} Student\\
\textbf{precondition} Das System ist online, der Student ist eingeloggt und es existiert mindestens ein Workspace.\\
\textbf{main flow} Der Student wählt bei einem Workspace die Aktion "`Workspace bearbeiten"'. Der Anwender hat nun die Möglichkeit den Workspace nach seinen Wünschen zu benennen und kann Widgets zu dem Workspace hinzufügen (extension point: Widget hinzufügen).\\
\textbf{postcondition} Der Workspace wurde nach den Vorstellungen des Studenten angepasst.
 
\textbf{extend relationship}\\
\textbf{base} "`Workspace hinzufügen"'\\
\textbf{extension point} Workspace bearbeiten\\
\textbf{extension} "`Workspace bearbeiten"'
 
\section*{Workspace löschen}
\textbf{use case} \emph{Workspace löschen}\\
\textbf{actors} Student\\
\textbf{precondition} Das System ist online, der Student ist eingeloggt und es existiert mindestens ein Workspace.\\
\textbf{main flow} Der Student wählt bei einem Workspace die Aktion "`Workspace löschen"'. Es erscheint eine Rückfrage, welche eine Bestätigung des Löschvorganges (mitsamt aller Widgets) erfragt. Bei positiver Rückmeldung gibt das System die Nachricht des Löschens aus. \\
\textbf{postcondition} Der Workspace und alle seine Widgets sind aus dem System entfernt.


\section*{Widget hinzufügen}
\textbf{use case} \emph{Widget hinzufügen}\\
\textbf{actors} Student\\
\textbf{precondition} Das System ist online, der Student ist eingeloggt und es existiert mindestens ein Workspace.\\
\textbf{main flow} Der Student befindet sich in einem Workspace und wählt die Aktion "`Widget hinzufügen"' Es erscheint eine Maske in der die zur Verfügung stehenden Widgets ausgewählt werden können. Der Student wählt das gewünschte Widget und fügt es dem Workspace hinzu. Wenn gewünscht kann der Student die Liste der Widgets über Suchfilter einschränken (extension point: Widgets filtern).\\
\textbf{postcondition} Das Widget wurde dem Workspace hinzugefügt.

\textbf{extend relationship}\\
\textbf{base} `Workspace bearbeiten"'\\
\textbf{extension point} Widget hinzufügen\\
\textbf{extension} "`Widget hinzufügen"'

\section*{Widgets filtern}
\textbf{use case} \emph{Widgets filtern}\\
\textbf{actors} Student\\
\textbf{precondition} Der Student ist dabei einem Workspace ein Widget hinzuzufügen.\\
\textbf{main flow} Der Student gibt in einem Textfeld eine Zeichenkette an, nach der im Widget-Namen gesucht wird. Des Weiteren kann er in einem binären Filter wählen, ob er nur Widgets angezeigt bekommen möchte, die in der Lage sind in einem Offline-Modus zu arbeiten.\\
\textbf{postcondition} In der Liste der zur Auswahl stehenden Widgets werden nur noch diejenigen angezeigt, die der Filterung entsprechen.
 
\textbf{extend relationship}\\
\textbf{base} "`Widget hinzufügen"'\\
\textbf{extension point} Widgets filtern\\
\textbf{extension} "`Widgets filtern"'
 
\section*{Widget löschen}
\textbf{use case} \emph{Widget löschen}\\
\textbf{actors} Student\\
\textbf{precondition} Das System ist online, der Student ist eingeloggt, er und es existiert mindestens ein Widget auf einem Workspace.\\
\textbf{main flow} Der Student befindet sich in einem Workspace und wählt bei einem Widget die Aktion "`Widget löschen"'. Es erscheint eine Rückfrage, welche eine Bestätigung des Löschvorganges erfragt. Bei positiver Rückmeldung gibt das System die Nachricht des Löschens aus.\\
\textbf{postcondition} Das Widget wurde aus dem System entfernt.
 
\section*{Anordnung der Widgets ändern}
\textbf{use case} \emph{Anordnung der Widgets ändern}\\
\textbf{actors} Student\\
\textbf{precondition} Das System ist online, der Student ist eingeloggt und befindet sich auf der Seite eines Work"-spaces mit mindestens zwei Widgets.\\
\textbf{main flow} Der Student hat die Möglichkeit die einzelnen Widgets innerhalb eines Work"-spaces über einen Drag and Drop Mechanismus neu anzuordnen. Er wählt hierfür ein Widget mit der Maus aus und zieht es an die gewünschte Position.\\
\textbf{postcondition} Die Anordnung der Widgets innerhalb des Work"-spaces hat sich nach dem Wunsch des Studenten geändert.
 
\section*{Informationen über Widgets erhalten}
\textbf{use case} \emph{Informationen über Widgets erhalten}\\
\textbf{actors} Student\\
\textbf{precondition} Der Student ist eingeloggt und bewegt sich zu einem Workspace.\\
\textbf{main flow} Die Widgets des Work"-spaces aktualisieren sich mit ihren Services. \\
\textbf{postcondition} Jedes Widget zeigt ihm die wichtigsten Informationen an. Hierzu gehören, wie viele neue Einträge es gibt, wie viele eventuell noch nicht mit dem Backend synchronisiert sind und ob das Widget online oder offline ist.
 
\section*{Überblick über Workspace/Widgets verschaffen}
\textbf{use case} \emph{Überblick verschaffen}\\
\textbf{actors} Student\\
\textbf{precondition} Der Student ist eingeloggt.\\
\textbf{main flow} Der Student wählt eine Aktion aus, die ihn zu einer Übersichtsseite innerhalb des Systems bringt.\\
\textbf{postcondition} Der Student befindet sich auf einer Seite, welche ihm seine Work"-spaces und Widgets anzeigt und ihm die wichtigsten Informationen darüber vermittelt.

\section*{Erkennen des Onlinestatus}
\textbf{use case} \emph{Erkennen des Onlinestatus}\\
\textbf{actors} Student\\
\textbf{precondition} Der Student ist eingeloggt, das System ist online.\\
\textbf{main flow} Der Student sieht auf den ersten Blick, dass das System online ist. Sowohl die Hauptapplikation, als auch die Widgets zeigen ihm dies an. Der Nutzer beendet die Internetverbindung.\\
\textbf{postcondition} Das System zeigt dem Nutzer den Verlust der Verbindung direkt an.
 
\section*{Starten der Anwendung ohne Internetverbindung}
\textbf{use case} \emph{Starten der Anwendung ohne Internetverbindung}\\
\textbf{actors} Student\\
\textbf{precondition} Es besteht keine Internetverbindung.\\
\textbf{main flow} Der Anwender öffnet seinen Browser und gibt die URL der PLE ein.\\
\textbf{postcondition} Das System hat sich trotz fehlender Internetverbindung geöffnet und stellt dem Studenten das ihm bekannte User-Interface dar.
 
\section*{Arbeiten mit Widgets ohne Internetverbindung}
\textbf{use case} \emph{Arbeiten mit Widgets ohne Internetverbindung}\\
\textbf{actors} Student\\
\textbf{precondition} Das System ist geladen, hat aber keine Internetverbindung.\\
\textbf{main flow} Der Anwender arbeitet mit den Widgets und kann neue Einträge etc. hinzufügen und bearbeiten.\\
\textbf{postcondition} Die Änderungen wurden zwischengespeichert. Dies wird dem Nutzer angezeigt.

\section*{Herstellen einer Internetverbindung}
\textbf{use case} \emph{Herstellen einer Internetverbindung}\\
\textbf{actors} Student\\
\textbf{precondition} Das System ist offline. Der Student hat in den Widgets Einträge hinzugefügt, bearbeiten oder gelöscht.\\
\textbf{main flow} Der Student stellt eine Verbindung mit dem Internet her.\\
\textbf{postcondition} Die offline vorgenommenen Arbeiten wurden mit dem jeweiligen Backend synchronisiert.