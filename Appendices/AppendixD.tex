\chapter{Anwendungszenario} \label{AppendixD}

\section*{Anwendungsszenario}
Im Folgenden werden die unterschiedlichen Anwendungsszenarien mit dem System exemplarisch an Student 1 und an Student 3 dargelegt, um die Komplexität der Anforderungen an eine PLE zu verdeutlichen.

\subsection*{Grundsätzliches Anwendungsszenario für Studenten zur Arbeit mit der PLE}
Um mit der PLE arbeiten zu können müssen die Studenten als erstes online eine Account in dem PLE-System erstellen. Anschließend melden sie sich mit ihren gewählten Login-Daten an. Bei erfolgreicher Anmeldung hat der Anwender die Möglichkeit direkt Work"-spaces zu erstellen, mit diesen zu arbeiten (Workspace umbenennen, Einstellungen vornehmen, Widgets hinzufügen etc).

\subsection*{Anwendungsszenario Student 1}
\begin{itemize}
 \item \emph{Tag 1:} Student 1 befindet sich in der Universität und verbindet seinen USB-Stick mit einem PC. Auf diesem Stick befindet sich die ausführbare mobile Version eines aktuellen Browsers. Er öffnet die Applikation online in diesem Browser (alle folgenden Aktionen werden mit dem selben Browser durchgeführt). Student 1 erstellt einen neuen Workspace und benennt ihn in "`PLE"' um. Anschließend sucht er die Widgets für den PLE Workspace aus der Widget-Datenbank heraus. Das System zeigt ihm dabei an, für welche Widgets eine Offline-Fähigkeit zur Verfügung steht. Daraufhin organisiert er die Anordnung der Widgets nach seinen Vorstellungen per Drag and Drop neu. Damit er mit den einzelnen Widgets auch arbeiten kann, meldet sich Student 1 schließlich bei den jeweiligen Services mit seinen Account-Daten an. Bevor er den Browser schließt haben sich alle Widgets mit ihren Services synchronisiert und zeigen ihm dies auch an.
 \item \emph{Tag 2:} Student 1 loggt sich an der Uni in das PLE-System ein. Das System zeigt ihm auf der Startseite (dem Dashboard) an wie viele neue Einträge es auf seinem PLE Workspace gibt. Ein direkter Link führt ihn zum Workspace. Er sieht, dass momentan mehrere Leute im Chat sind und unterhält sich über das Widget mit ihnen. Der Dozent hat einige globale Todos angelegt und die ersten Nachrichten kommen über den Twitter Kanal herein. Er sieht, dass Student 3 einen längeren Text im Chat geschrieben hat, beschließt diesen jedoch später zu lesen. Gleiches gilt für den Einführungsartikel des Dozenten im Kursblog. Hierfür erstellt er sich Einträge in seiner Todo-Liste. Abschließend schließt Student 1 den Browser und entfernt den USB-Stick.
 \item \emph{Tag 3:} Student 1 öffnet die Applikation in seinem mobilen Browser zu Hause. Das System erkennt, dass es sich im Offline-Modus befindet und versucht keine Synchronisierung mit dem Internet herzustellen. Student 1 liest den Text, den Student 3 im Chat hinterlegt hat und beantwortet die darin gestellten Fragen ebenfalls im Chatfenster. Anschließend teilt er dies über Twitter mit und erledigt seinen Todo-Eintrag.
 \item \emph{Tag 4:} Student 1 loggt sich im Internet-Cafe mit seinem mobilen Browser in das System ein. Das System erkennt, dass es online ist, lädt die neuesten Einträge der Services herunter und synchronisiert die nur lokal vorgenommenen Aktionen (Twitter, Chat, Todo). Per Mail hat Student 1 den Vorschlag von Student 2 bekommen gemeinsam mit Student 4 eine Lerngruppe zum Thema Datenbanken zu gründen. Hierzu wollen sie unter anderem ein Chat-System benutzen. Student 1 schlägt per Mail das ihm bekannte Chat-Widget für die PLE-Applikation vor. Er erstellt einen neuen Workspace, nennt ihn “Lerngruppe DB” und fügt das Chat-Widget mit den passenden Einstellungen hinzu.
 \item \emph{Tag 5:} Student 1 sieht in seinem Dashboard, dass es für den Workspace “Lerngruppe DB” 12 neue Einträge gibt. Er geht direkt zu dem Workspace und stellt fest, dass die beiden anderen Studenten sich ebenfalls in dem Chat angemeldet haben. Da momentan alle online verfügbar sind, beginnen sie ihre erste Gruppenunterhaltung und planen die weitere Vorgehensweise.
\end{itemize}

\subsection*{Anwendungsszenario Student 3}
Student 3 nutzt das System hauptsächlich mit seinem Smartphone, welches in der Lage ist sich über WLAN sowie UMTS mit dem Internet zu verbinden.

\begin{itemize}
 \item \emph{Tag 1:} Student 3 meldet sich ähnlich wie Student 1 in der PLE an und erstellt für seine Uni-Kurse jeweils einen Workspace. Außerdem legt er sich einen Workspace an, in dem sich nicht kursspezifische Widgets finden. Hierzu gehören sein RSS-Reader, eine Todo Liste sowie ein Google-Calendar-Widget.
 \item \emph{Tag 2:} Student 3 verbindet sich zu Hause mit seinem WLAN und loggt sich in der PLE ein. Nach Prüfung des Dashboards erstellt er sich in der allgemeinen Todo Liste für den Tag. Er sieht, dass es in seinem RSS-Reader 9 neue Artikel gibt. Diese möchte er auf seinem Weg zur Uni in der U-Bahn lesen. Auf dem Weg in die Universität ruft Student 3 das System auf seinem Smartphone auf. Da kein Internetzugang besteht greift das System nur auf die lokalen Daten zurück. Das Calendar-Widget in seinem globalen Workspace ist nicht offlinefähig. Aus diesem Grund wird es auch nur ausgegraut und nicht-funktional angezeigt. Der Student ist jedoch in der Lage seine lokal gecachten RSS-Artikel zu lesen. Bei Lektüre eines Artikels über Javascript fällt ihm ein, dass er sein Uni-Projekt noch mit der neuesten jQuery Version aktualisieren wollte. Hierzu erstellt er sich ein weiteres Todo auf seiner Liste.
 In der Uni verbindet er sein Smartphone mit dem Uni-WLAN. Der RSS-Reader markiert die 6 Artikel die, Student 3 in der U-Bahn gelesen hat als gelesen und synchronisiert das neueste Todo-Eintrag mit dem Service im Internet.  
\end{itemize}
