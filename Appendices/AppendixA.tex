\chapter{Anhang} \label{AppendixA}

\section{Inhalt} 
Der Anhang besteht aus vier Teilen: Der Teil \nameref{AppendixB} beschreibt die Installation von Plesynd und Wookie. Der Teil \nameref{AppendixC} beschreibt die wichtigsten Komponenten von Plesynd, um zukünftigen Arbeiten, die auf dieser Arbeit aufbauen einen besseren Überblick zu gewähren. Der Anhang \nameref{AppendixD} beschreibt ausführlich ein mögliches Anwendungszenario für eine PLE und der Anhang \nameref{AppendixE} bringt die in Kapitel \ref{chapter:Kapitel3} vorgestellten Anwendungsfälle in die Form einer technischen Dokumentation.

\section{Inhalt der CD}
Angefügt an die Arbeit befindet sich eine CD mit dem entwickelten Plesynd-System. Die Verzeichnisstruktur ist wie folgt aufgebaut:
\begin{itemize}
 \item \texttt{plesynd/} enthält Plesynd
 \item \texttt{todo/} enthält das Todo-Widget als gepacktes W3C-Widget
 \item \texttt{docs/} enthält die automatisch generierte Quellcodedokumentation
 \item \texttt{thesis/} enthält die vorliegende Arbeit im Latex und im PDF Format
\end{itemize}

\section{Online}
Alle für diese Arbeit produzierten Inhalte sind online frei verfügbar zu finden:
\begin{itemize}
 \item Die Arbeit selbst: \url{https://github.com/rommsen/thesis/}
 \item Plesynd: \url{https://github.com/rommsen/plesynd}
 \item Plesynd-Todo: \url{https://github.com/rommsen/plesynd-todo/}
\end{itemize}


