\chapter{Anhang} \label{AppendixA}

\section{Inhalt} 
Der Anhang besteht aus zwei Teilen: Der Teil \nameref{AppendixB} beschreibt die Installation von Plesynd und Wookie. Der Teil \nameref{AppendixC} beschreibt die wichtigsten Komponenten von Plesynd, um zukünftigen Arbeiten, die auf dieser Arbeit aufbauen einen besseren Überblick zu gewähren

\section{Inhalt der CD}
Angefügt an die Arbeit befindet sich eine CD mit dem entwickelten Plesynd-System. Die Verzeichnisstruktur ist wie folgt aufgebaut:
\begin{itemize}
 \item \texttt{thesis/} enthält die vorliegende Arbeit im Latex und im PDF Format
 \item \texttt{plesynd/} enthält Plesynd
 \item \texttt{todo/} enthält das Todo-Widget als gepacktes W3C-Widget
 \item \texttt{docs/} enthält die automatisch generierte Quellcodedokumentation
 \item \texttt{thesis/} enthält die vorliegende Arbeit im Latex und im PDF Format
\end{itemize}


entfernt:


\subsection{Anwendungsszenario}
Im Folgenden werden die unterschiedlichen Anwendungsszenarien mit dem System exemplarisch an Student 1 und an Student 3 dargelegt, um die Komplexität der Anforderungen an eine PLE zu verdeutlichen.

\subsubsection*{Grundsätzliches Anwendungsszenario für Studenten zur Arbeit mit der PLE}
Um mit der PLE arbeiten zu können müssen die Studenten als erstes online eine Account in dem PLE-System erstellen. Anschließend melden sie sich mit ihren gewählten Login-Daten an. Bei erfolgreicher Anmeldung hat der Anwender die Möglichkeit direkt Work"-spaces zu erstellen, mit diesen zu arbeiten (Workspace umbenennen, Einstellungen vornehmen, Widgets hinzufügen etc).

\subsubsection*{Anwendungsszenario Student 1}
\begin{itemize}
 \item \emph{Tag 1:} Student 1 befindet sich in der Universität und verbindet seinen USB-Stick mit einem PC. Auf diesem Stick befindet sich die ausführbare mobile Version eines aktuellen Browsers. Er öffnet die Applikation online in diesem Browser (alle folgenden Aktionen werden mit dem selben Browser durchgeführt). Student 1 erstellt einen neuen Workspace und benennt ihn in "`PLE"' um. Anschließend sucht er die Widgets für den PLE Workspace aus der Widget-Datenbank heraus. Das System zeigt ihm dabei an, für welche Widgets eine Offline-Fähigkeit zur Verfügung steht. Daraufhin organisiert er die Anordnung der Widgets nach seinen Vorstellungen per Drag and Drop neu. Damit er mit den einzelnen Widgets auch arbeiten kann, meldet sich Student 1 schließlich bei den jeweiligen Services mit seinen Account-Daten an. Bevor er den Browser schließt haben sich alle Widgets mit ihren Services synchronisiert und zeigen ihm dies auch an.
 \item \emph{Tag 2:} Student 1 loggt sich an der Uni in das PLE-System ein. Das System zeigt ihm auf der Startseite (dem Dashboard) an wie viele neue Einträge es auf seinem PLE Workspace gibt. Ein direkter Link führt ihn zum Workspace. Er sieht, dass momentan mehrere Leute im Chat sind und unterhält sich über das Widget mit ihnen. Der Dozent hat einige globale Todos angelegt und die ersten Nachrichten kommen über den Twitter Kanal herein. Er sieht, dass Student 3 einen längeren Text im Chat geschrieben hat, beschließt diesen jedoch später zu lesen. Gleiches gilt für den Einführungsartikel des Dozenten im Kursblog. Hierfür erstellt er sich Einträge in seiner Todo-Liste. Abschließend schließt Student 1 den Browser und entfernt den USB-Stick.
 \item \emph{Tag 3:} Student 1 öffnet die Applikation in seinem mobilen Browser zu Hause. Das System erkennt, dass es sich im Offline-Modus befindet und versucht keine Synchronisierung mit dem Internet herzustellen. Student 1 liest den Text, den Student 3 im Chat hinterlegt hat und beantwortet die darin gestellten Fragen ebenfalls im Chatfenster. Anschließend teilt er dies über Twitter mit und erledigt seinen Todo-Eintrag.
 \item \emph{Tag 4:} Student 1 loggt sich im Internet-Cafe mit seinem mobilen Browser in das System ein. Das System erkennt, dass es online ist, lädt die neuesten Einträge der Services herunter und synchronisiert die nur lokal vorgenommenen Aktionen (Twitter, Chat, Todo). Per Mail hat Student 1 den Vorschlag von Student 2 bekommen gemeinsam mit Student 4 eine Lerngruppe zum Thema Datenbanken zu gründen. Hierzu wollen sie unter anderem ein Chat-System benutzen. Student 1 schlägt per Mail das ihm bekannte Chat-Widget für die PLE-Applikation vor. Er erstellt einen neuen Workspace, nennt ihn “Lerngruppe DB” und fügt das Chat-Widget mit den passenden Einstellungen hinzu.
 \item \emph{Tag 5:} Student 1 sieht in seinem Dashboard, dass es für den Workspace “Lerngruppe DB” 12 neue Einträge gibt. Er geht direkt zu dem Workspace und stellt fest, dass die beiden anderen Studenten sich ebenfalls in dem Chat angemeldet haben. Da momentan alle online verfügbar sind, beginnen sie ihre erste Gruppenunterhaltung und planen die weitere Vorgehensweise.
\end{itemize}

\subsubsection*{Anwendungsszenario Student 3}
Student 3 nutzt das System hauptsächlich mit seinem Smartphone, welches in der Lage ist sich über WLAN sowie UMTS mit dem Internet zu verbinden.

\begin{itemize}
 \item \emph{Tag 1:} Student 3 meldet sich ähnlich wie Student 1 in der PLE an und erstellt für seine Uni-Kurse jeweils einen Workspace. Außerdem legt er sich einen Workspace an, in dem sich nicht kursspezifische Widgets finden. Hierzu gehören sein RSS-Reader, eine Todo Liste sowie ein Google-Calendar-Widget.
 \item \emph{Tag 2:} Student 3 verbindet sich zu Hause mit seinem WLAN und loggt sich in der PLE ein. Nach Prüfung des Dashboards erstellt er sich in der allgemeinen Todo Liste für den Tag. Er sieht, dass es in seinem RSS-Reader 9 neue Artikel gibt. Diese möchte er auf seinem Weg zur Uni in der U-Bahn lesen. Auf dem Weg in die Universität ruft Student 3 das System auf seinem Smartphone auf. Da kein Internetzugang besteht greift das System nur auf die lokalen Daten zurück. Das Calendar-Widget in seinem globalen Workspace ist nicht offlinefähig. Aus diesem Grund wird es auch nur ausgegraut und nicht-funktional angezeigt. Der Student ist jedoch in der Lage seine lokal gecachten RSS-Artikel zu lesen. Bei Lektüre eines Artikels über Javascript fällt ihm ein, dass er sein Uni-Projekt noch mit der neuesten jQuery Version aktualisieren wollte. Hierzu erstellt er sich ein weiteres Todo auf seiner Liste.
 In der Uni verbindet er sein Smartphone mit dem Uni-WLAN. Der RSS-Reader markiert die 6 Artikel die, Student 3 in der U-Bahn gelesen hat als gelesen und synchronisiert das neueste Todo-Eintrag mit dem Service im Internet.  
\end{itemize}


\subsection{Registrieren}
\textbf{use case} \emph{Registrieren}\\
\textbf{actors} Student\\
\textbf{precondition} Das System ist online, der Student ist nicht eingeloggt und es existiert mindestens ein Workspace.\\
\textbf{main flow} Der Student bekommt auf der Startseite die Möglichkeit sich für den Zugang zum System zu registrieren. Er füllt ein Formular mit seinen Daten (Vorname, Nachname, E-Mail-Adresse, Nutzername, Passwort) aus und schickt das Formular ab.\\
\textbf{postcondition} Der Student ist mit einer Nutzername-/Passwortkombination und seiner E-Mail-Adresse im System hinterlegt und kann sich somit in das System einloggen.\\
\textbf{exceptional flow} Nutzername oder Mail-Adresse bereits vorhanden. Nutzername und E-Mail-Adresse dürfen nur einmal im System vorkommen.\\
\textbf{postcondition} Der Student bekommt eine Fehlermeldung und muss seine eingetragenen Daten ändern.
 
\subsection{Login}
\textbf{use case} \emph{Login}\\
\textbf{actors} Student\\
\textbf{precondition} Das System ist online, der Student ist nicht eingeloggt, ist aber in dem System registriert.\\
\textbf{main flow} Der Student bekommt auf der Startseite die Möglichkeit sich im System anzumelden. Er füllt ein Formular mit seiner Nutzername-/Passwortkombination aus und schickt das Formular ab.\\
\textbf{postcondition} Der Student ist im System eingeloggt und das System präsentiert ihm eine Übersicht über seine aktuellen Work"-spaces und die wichtigsten aggregierten Informationen dieser.

\subsection{Logout}
\textbf{use case} \emph{Logout}\\
\textbf{actors} Student\\
\textbf{precondition} Das System ist online, der Student ist eingeloggt.\\
\textbf{main flow} Der Student wählt die Aktion "`Logout"' und wird anschließend auf die Login-Seite weitergeleitet.\\
\textbf{postcondition} Die aktuelle Session des Anwenders ist beendet und es ist ohne einen weiteren Login nicht möglich, auf die Daten des Anwenders zuzugreifen 

\subsection{Workspace hinzufügen}
\textbf{use case} \emph{Workspace hinzufügen}\\
\textbf{actors} Student\\
\textbf{precondition} Das System ist online, der Student ist eingeloggt.\\
\textbf{main flow} Der Student wählt die Aktion "`Workspace hinzufügen"'. Es öffnet sich hierdurch ein neuer Bereich, welcher einen neuen Workspace repräsentiert. Der Student hat nun die Möglichkeit den Workspace nach seinen Wünschen anzupassen (extension point: Workspace bearbeiten).\\
\textbf{postcondition} Ein neuer Workspace wurde dem System des Studenten hinzugefügt.
 
\subsection{Workspace bearbeiten}
\textbf{use case} \emph{Workspace bearbeiten}\\
\textbf{actors} Student\\
\textbf{precondition} Das System ist online, der Student ist eingeloggt und es existiert mindestens ein Workspace.\\
\textbf{main flow} Der Student wählt bei einem Workspace die Aktion "`Workspace bearbeiten"'. Der Anwender hat nun die Möglichkeit den Workspace nach seinen Wünschen zu benennen und kann Widgets zu dem Workspace hinzufügen (extension point: Widget hinzufügen).\\
\textbf{postcondition} Der Workspace wurde nach den Vorstellungen des Studenten angepasst.
 
\textbf{extend relationship}\\
\textbf{base} "`Workspace hinzufügen"'\\
\textbf{extension point} Workspace bearbeiten\\
\textbf{extension} "`Workspace bearbeiten"'
 
\subsection{Workspace löschen}
\textbf{use case} \emph{Workspace löschen}\\
\textbf{actors} Student\\
\textbf{precondition} Das System ist online, der Student ist eingeloggt und es existiert mindestens ein Workspace.\\
\textbf{main flow} Der Student wählt bei einem Workspace die Aktion "`Workspace löschen"'. Es erscheint eine Rückfrage, welche eine Bestätigung des Löschvorganges (mitsamt aller Widgets) erfragt. Bei positiver Rückmeldung gibt das System die Nachricht des Löschens aus. \\
\textbf{postcondition} Der Workspace und alle seine Widgets sind aus dem System entfernt.


\subsection{Widget hinzufügen}
\textbf{use case} \emph{Widget hinzufügen}\\
\textbf{actors} Student\\
\textbf{precondition} Das System ist online, der Student ist eingeloggt und es existiert mindestens ein Workspace.\\
\textbf{main flow} Der Student befindet sich in einem Workspace und wählt die Aktion "`Widget hinzufügen"' Es erscheint eine Maske in der die zur Verfügung stehenden Widgets ausgewählt werden können. Der Student wählt das gewünschte Widget und fügt es dem Workspace hinzu. Wenn gewünscht kann der Student die Liste der Widgets über Suchfilter einschränken (extension point: Widgets filtern).\\
\textbf{postcondition} Das Widget wurde dem Workspace hinzugefügt.

\textbf{extend relationship}\\
\textbf{base} `Workspace bearbeiten"'\\
\textbf{extension point} Widget hinzufügen\\
\textbf{extension} "`Widget hinzufügen"'

\subsection{Widgets filtern}
\textbf{use case} \emph{Widgets filtern}\\
\textbf{actors} Student\\
\textbf{precondition} Der Student ist dabei einem Workspace ein Widget hinzuzufügen.\\
\textbf{main flow} Der Student gibt in einem Textfeld eine Zeichenkette an, nach der im Widget-Namen gesucht wird. Des Weiteren kann er in einem binären Filter wählen, ob er nur Widgets angezeigt bekommen möchte, die in der Lage sind in einem Offline-Modus zu arbeiten.\\
\textbf{postcondition} In der Liste der zur Auswahl stehenden Widgets werden nur noch diejenigen angezeigt, die der Filterung entsprechen.
 
\textbf{extend relationship}\\
\textbf{base} "`Widget hinzufügen"'\\
\textbf{extension point} Widgets filtern\\
\textbf{extension} "`Widgets filtern"'
 
\subsection{Widget löschen}
\textbf{use case} \emph{Widget löschen}\\
\textbf{actors} Student\\
\textbf{precondition} Das System ist online, der Student ist eingeloggt, er und es existiert mindestens ein Widget auf einem Workspace.\\
\textbf{main flow} Der Student befindet sich in einem Workspace und wählt bei einem Widget die Aktion "`Widget löschen"'. Es erscheint eine Rückfrage, welche eine Bestätigung des Löschvorganges erfragt. Bei positiver Rückmeldung gibt das System die Nachricht des Löschens aus.\\
\textbf{postcondition} Das Widget wurde aus dem System entfernt.
 
\subsection{Anordnung der Widgets ändern}
\textbf{use case} \emph{Anordnung der Widgets ändern}\\
\textbf{actors} Student\\
\textbf{precondition} Das System ist online, der Student ist eingeloggt und befindet sich auf der Seite eines Work"-spaces mit mindestens zwei Widgets.\\
\textbf{main flow} Der Student hat die Möglichkeit die einzelnen Widgets innerhalb eines Work"-spaces über einen Drag and Drop Mechanismus neu anzuordnen. Er wählt hierfür ein Widget mit der Maus aus und zieht es an die gewünschte Position.\\
\textbf{postcondition} Die Anordnung der Widgets innerhalb des Work"-spaces hat sich nach dem Wunsch des Studenten geändert.
 
\subsection{Informationen über Widgets erhalten}
\textbf{use case} \emph{Informationen über Widgets erhalten}\\
\textbf{actors} Student\\
\textbf{precondition} Der Student ist eingeloggt und bewegt sich zu einem Workspace.\\
\textbf{main flow} Die Widgets des Work"-spaces aktualisieren sich mit ihren Services. \\
\textbf{postcondition} Jedes Widget zeigt ihm die wichtigsten Informationen an. Hierzu gehören, wie viele neue Einträge es gibt, wie viele eventuell noch nicht mit dem Backend synchronisiert sind und ob das Widget online oder offline ist.
 
\subsection{Überblick über Workspace/Widgets verschaffen}
\textbf{use case} \emph{Überblick verschaffen}\\
\textbf{actors} Student\\
\textbf{precondition} Der Student ist eingeloggt.\\
\textbf{main flow} Der Student wählt eine Aktion aus, die ihn zu einer Übersichtsseite innerhalb des Systems bringt.\\
\textbf{postcondition} Der Student befindet sich auf einer Seite, welche ihm seine Work"-spaces und Widgets anzeigt und ihm die wichtigsten Informationen darüber vermittelt.

\subsection{Erkennen des Onlinestatus}
\textbf{use case} \emph{Erkennen des Onlinestatus}\\
\textbf{actors} Student\\
\textbf{precondition} Der Student ist eingeloggt, das System ist online.\\
\textbf{main flow} Der Student sieht auf den ersten Blick, dass das System online ist. Sowohl die Hauptapplikation, als auch die Widgets zeigen ihm dies an. Der Nutzer beendet die Internetverbindung.\\
\textbf{postcondition} Das System zeigt dem Nutzer den Verlust der Verbindung direkt an.
 
\subsection{Starten der Anwendung ohne Internetverbindung}
\textbf{use case} \emph{Starten der Anwendung ohne Internetverbindung}\\
\textbf{actors} Student\\
\textbf{precondition} Es besteht keine Internetverbindung.\\
\textbf{main flow} Der Anwender öffnet seinen Browser und gibt die URL der PLE ein.\\
\textbf{postcondition} Das System hat sich trotz fehlender Internetverbindung geöffnet und stellt dem Studenten das ihm bekannte User-Interface dar.
 
\subsection{Arbeiten mit Widgets ohne Internetverbindung}
\textbf{use case} \emph{Arbeiten mit Widgets ohne Internetverbindung}\\
\textbf{actors} Student\\
\textbf{precondition} Das System ist geladen, hat aber keine Internetverbindung.\\
\textbf{main flow} Der Anwender arbeitet mit den Widgets und kann neue Einträge etc. hinzufügen und bearbeiten.\\
\textbf{postcondition} Die Änderungen wurden zwischengespeichert. Dies wird dem Nutzer angezeigt.

\subsection{Herstellen einer Internetverbindung}
\textbf{use case} \emph{Herstellen einer Internetverbindung}\\
\textbf{actors} Student\\
\textbf{precondition} Das System ist offline. Der Student hat in den Widgets Einträge hinzugefügt, bearbeiten oder gelöscht.\\
\textbf{main flow} Der Student stellt eine Verbindung mit dem Internet her.\\
\textbf{postcondition} Die offline vorgenommenen Arbeiten wurden mit dem jeweiligen Backend synchronisiert.