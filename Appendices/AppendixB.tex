% Appendix B

\chapter{Installation / Deployment von Plesynd}
\label{AppendixB}
\lhead{Anhang B \emph{Installation / Deployment von Plesynd}}

\section{Installation}
Es wird davon ausgegangen, dass PHP (mindestens Version 5.3), ein Webserver wie Apache, Mysql (mindestens Version 5.1), Java und Ant bereits auf dem System, auf dem Plesynd installiert werden soll, eingerichtet wurden.

\subsection{Wookie}
Wookie muss wie in \url{http://wookie.apache.org/docs/source.html} beschrieben installiert werden. Anschließend kann der Server mit \texttt{ant run} im Wookie Verzeichnis gestartet werden. Standardmäßig hört Wookie auf \path{http://localhost:8080/wookie}. Um einen anderen Virtual Host zu benutzen muss die Datei \texttt{local.widgetserver.properties} angepasst und direkt im Wookie Verzeichnis abgelegt werden (siehe \url{http://wookie.apache.org/docs/developer/running.html}). Hier sollte ein anderer Port gewählt werden als der auf den der Apache Server hört.

\subsection{Plesynd}
Für die Installation von Plesynd müssen mehrere Schritte durchgeführt werden:
\begin{enumerate}
 \item Composer installieren: \url{http://getcomposer.org/}
 \item Node.js installieren: \texttt{https://github.com/joyent/node/}\\
 \texttt{wiki/Installing-Node.js-via-package-manager}
 \item Arbeitskopie aus github auschecken:\\ \texttt{git clone git@github.com:rommsen/plesynd.git}\\
 \texttt{verzeichnis\_in\_dem\_plesynd\_installiert\_werden\_soll}
 \item Symfony2 installieren: \texttt{./composer install} im Verzeichnis der Arbeitskopie
 \item Virtual Host im Webserver anlegen und Document Root auf \\
 \texttt{plesynd\_verzeichnis/web} setzen.
 \item Configs anpassen: In \texttt{app/config} die Datei parameters.yml.dist in parameters.yml umbenennen und die Einstellungen anpassen: 
 \begin{lstlisting}
 parameters:
    database_driver:   pdo_mysql
    database_host:     localhost
    database_port:     ~
    database_name:     plesynd
    database_user:     [Datenbank-User]
    database_password: [Datenbank-Password]

    mailer_transport:  smtp
    mailer_host:       smtp.ipark-media.de
    mailer_user:       [User für Mailversand]
    mailer_password:   [Password für Mailversand]
    mailer_encryption: null
    mailer_auth_mode:  plain
    mailer_port:       25
    delivery_address:  [Wenn gesetzt, gehen alle Mails an diese Adresse]

    locale:            en
    secret:            [Zufaelliger String, AENDERN]

    plesynd_protocol:  [z.B. http://] 
    plesynd_host:      [z.B. plesynd]

    wookie_protocol:   [z.B. http://
    wookie_host:       [z.B. 127.0.0.1
    wookie_port:       [z.B. 8080]
    wookie_path:       [z.B. wookie/]
    wookie_api_key:    [z.B. TEST, kann in Wookie hinterlegt werden]
    wookie_shared_data_key: [z.B. shared_data]
    wookie_login_name: [z.B. user]
 \end{lstlisting}

 \item Datenbank erstellen: \texttt{./app/console doctrine:database:create} im Verzeichnis der Arbeitskopie
 \item Datenbankschema erstellen: \texttt{./app/console doctrine:schema:update --force} im Verzeichnis der Arbeitskopie
 \item Wenn der Virtual Host auf z.B. auf http://plesynd hört, kann das System unter \path{http://plesynd/app.php} im Produktivmodus geöffnet werden. Für den Entwicklungsmodus kann \path{http://plesynd/app_dev.php} aufgerufen werden.   
\end{enumerate}

\subsection{Todo-Widget}
\begin{enumerate}
 \item Arbeitskopie aus github auschecken: \\
 \texttt{git clone git@github.com:rommsen/plesynd-todo.git}\\ 
 \texttt{verzeichnis\_in\_dem\_das\_widget\_hinterlegt\_werden\_soll}
 \item Pfad von \texttt{wookie.root.dir} in \texttt{build.xml} auf das Installationsverzeichnis von Wookie anpassen.
 \item Pfad von \texttt{widget.root.dir} in \texttt{build.xml} auf den aktuellen Widgetordner anpassen.
\end{enumerate}

\section{Deployment}

\subsection{Plesynd}
Wenn bei den Konsolenkommandos der Parameter --env gesetzt wird, bestimmt dies ob diese für die Entwicklungsgebung (\texttt{--env=dev}) oder die Produktivumgebung (\texttt{--env=prod}) ausgeführt werden. In der Entwicklungsgebung stehen zusätzliche Loggingmechanismen etc. zur Verfügung.
\begin{enumerate}
 \item Leeren des Caches \texttt{./app/console cache:clear --env=prod|dev}
 \item Aufwärmen des Caches \texttt{./app/console cache:warmup --env=prod|dev}
 \item Dumpen der Assets (Javascript-, CSS-Dateien) in ein Verzeichnis, auf das der Webserver zugreifen kann \texttt{./app/console assetic:dump --env=prod|dev}. Zusätzlich kann der Parameter \texttt{--watch} übergeben werden, damit die Assets bei Änderung der Datei erneut gedumpt werden.
\end{enumerate}
\subsection{Todo-Widget}
\begin{enumerate}
 \item Nachdem Wookie über \texttt{ant run} gestartet wurde kann das Widget aus dem Widget-Verzeichnis heraus über \texttt{ant deploy-widget -Dwidget.shortname=todo} in der Wookie Instanz hinterlegt werden.
\end{enumerate}